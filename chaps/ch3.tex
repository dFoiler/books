% !TEX root = ../main.tex

\documentclass[../main.tex]{subfiles}

\begin{document}

\section{Whittaker models}
Recall that we have fixed a non-unit character $\psi$ of $K^+$, identified with a character of $U$ and found that $\pi\coloneqq\op{Ind}^P_U$ is an irreducible representation of dimension $q-1$. If $\chi$ is a character of $G$, then by the Frobenius reciprocity theorem,
\[\left(\chi,\op{Ind}^G_U\psi\right)_G=\left(\chi,\op{Ind}^G_B\pi\right)_G=\left(\op{Res}^G_B\chi,\pi\right)_B=0.\]
Hence, all the irreducible components of $\op{Ind}^G_U\psi$ are of dimension $>1$. Each one of them appears in multiplicity $1$ by \Cref{prop:ind-psi-has-no-mult}. On the other hand, if we sum up the dimensions of all higher-dimensional representations of $G$, we find that it is equal to $\dim\op{Ind}^G_U\psi=(q-1)^2(q+1)$. This follows from the table in \cref{sec:classify-cusp} and from the following identity:
\[\frac12\left(q^2-q\right)(q-1)+(q-1)q+\frac12(q-1)(q-2)(q+1)=(q-1)^2(q+1).\]
We conclude therefore the following.
\begin{theorem} \label{thm:ind-psi-has-reps}
	The representation $\op{Ind}^G_U\psi$ is equal to the direct sum of all higher-dimensional irreducible representations of $G$, each with multiplicity $1$.
\end{theorem}
Note that the same conclusion can be drawn from the following lemma, which his proved, however, without explicit use of the finiteness of $K$.
\begin{lemma} \label{lem:ind-psi-has-reps}
	Let $\rho$ be an irreducible representation of $G$ of dimension $>1$. Then the following are true.
	\begin{listalph}
		\item $\op{Res}^G_PV_\rho=\op{Res}^B_PJ(V_\rho)\oplus V_\pi$;
		\item $\dim J(V_\rho)=\dim\rho-(q-1)$;
		\item The multiplicity of $\rho$ in $\op{Ind}^G_U\psi$ is equal to $1$.
	\end{listalph}
\end{lemma}
\begin{proof}
	If $\dim\rho=q-1$, i.e., if $\rho$ is cuspidal, then $J(V_\rho)=0$, and $\op{Res}^G_PV_\rho=V_\pi$ by \Cref{prop:cuspidal-dim}. Formula (a) is therefore true in this case.

	If $\dim\rho\ge q$, i.e., if $\rho$ is non-cuspidal, then $J(V_\rho)\ne0$ and there exists a character $\mu$ of $B$ and a representation $\rho'$ of $G$ such that $\widehat\mu=\op{Ind}^G_P\mu=\rho'\oplus\rho$ and $\dim\rho'\le1$. Further, by \Cref{lem:decomp-res-mu-hat}, we have $\op{Res}^G_PV_{\widehat\mu}=\op{Res}^B_PJ(V_{\widehat\mu})\oplus V_\pi$; hence,
	\begin{equation}
		\op{Res}^G_PV_{\rho'}\oplus\op{Res}^G_PV_\rho=\op{Res}^G_PJ(V_{\rho'})\oplus\op{Res}^G_PJ(V_\rho)\oplus V_\pi. \label{eq:almost-res-rho-decomp}
	\end{equation}
	Now, $V_\pi$ does not contain any one-dimensional $P$-modules because $\dim V_\pi=q-1>1$ and $V_\pi$ is $P$-irreducible. It follows that $V_\pi\subseteq\op{Res}^G_PV_\rho$. Further, $\op{Res}^G_PJ(V_\rho)$ is certainly contained in $V_\rho$, and by \Cref{lem:eigens-of-jac-mu-hat}, it decomposes into a direct sum of one-dimensional $P$-subspaces. Hence, $\op{Res}^B_PV_\pi\cap\op{Res}^B_PJ(V_\rho)=0$, and thus the right-hand side of (a) is contained in its left-hand side. If $\dim\rho=q$, then because the dimension of the right-hand side of (a) is $\ge1+(q-1)$, we may conclude the equality. If $\dim\rho=q+1$, then $\rho'=0$, and \eqref{eq:almost-res-rho-decomp} coincides with (a).

	Note that we have also proved that the multiplicity of $\pi$ in $\op{Res}^G_P\rho$ is $1$. Hence, by the Frobenius reciprocity theorem the multiplicity of $\rho$ in $\op{Ind}^G_U\psi=\op{Ind}^G_B\pi$ is $1$.
\end{proof}

Recall that $\op{Ind}^G_UV_\psi$ can be identified with the space of all functions $F\colon G\to\CC$ such that
\[F(ug)=\psi(u)F(g)\qquad\text{for all }u\in U\text{ and }g\in G.\]
The group $G$ operates on $\op{Ind}^G_UV_\psi$ by the following law:
\[(sF)(g)=F(gs).\]
If $\rho$ is now an irreducible higher-dimensional representation of $G$, then $V_\rho$ can be embedded in $\op{Ind}^G_UV_\psi$. For every $v\in V_\rho$, there exist therefore a function $W_v\colon G\to\CC$ called a \textit{Whittaker function} of $\rho$ such that the following rules hold:
\[\begin{array}{cc}
	W_v=0\iff v=0, \\
	W_{c_1v_1+c_2v_2}=c_1W_{v_1}+c_2W_{v_2} & \text{for }c_1,c_2\in\CC, \\
	W_v(ug)=\psi(u)W_v(g) & \text{for }u\in U\text{ and }g\in G, \\
	W_{\rho(s)v}=W_v(gs) & \text{for }s,g\in G.
\end{array}\]
The set of all functions $W_v$ forms a $G$-subspace $W(\rho)$ of $\op{Ind}^G_UV_\psi$ called the \textit{Whittaker model} of $\rho$. By \Cref{thm:ind-psi-has-reps} or \Cref{lem:ind-psi-has-reps}, this subspace is uniquely determined within $\op{Ind}^G_UV_\psi$. Moreover, if $\rho'$ is an additional higher-dimensional representation of $G$, then $W(\rho)\cap W(\rho')=0$.

\section{The \texorpdfstring{$\Gamma$}{ Gamma}-function of a representation}
Let $\rho$ be a higher-dimensional irreducible representation of $G$. The $P$-decomposition
\begin{equation}
	V_\rho=J(V_\rho)\oplus V_\pi \label{eq:decomp-res-rho}
\end{equation}
of $V_\rho$, obtained in \Cref{lem:ind-psi-has-reps}, is also an $A$-decomposition, where we recall that
\[A\coloneqq\left\{\begin{bmatrix}
	\alpha & 0 \\
	0 & 1
\end{bmatrix}:\alpha\in K^\times\right\}\]
is the subgroup of $P$ which is canonically isomorphic to $K^\times$. If $\dim\rho\geq$, then $\rho=\rho_{(\mu_1,\mu_2)}$ are characters of $K^\times$. By \Cref{lem:eigens-of-jac-mu-hat}, they are the eigenvalues of $A$ on $J(V_\rho)$. If $\dim\rho=q+1$, then $\mu_1\ne\mu_2$ and $\dim J(V_\rho)=2$. If $\dim\rho=q$, then $\mu_1=\mu_2$ and $\dim J(V_\rho)=1$. If $\dim\rho=q-1$, then $J(V_\rho)=0$. In the first two cases we call $\mu_1^{-1}$, $\mu_2^{-1}$, and $\mu_1^{-1}$, respectively, the \textit{exceptional characters} for $\rho$. In the third case there are no exceptional characters. In any case, if the inverse of a character $\omega$ of $K^\times$ is not exceptional, then it is not an eigenvalue of $A$ operating on $J(V_\rho)$ through $\rho$.
\begin{lemma} \label{lem:almost-gamma-func}
	If a character $\omega$ of $K^\times$ is not an exceptional character for $\rho$, then any two linear functionals $\ell_1$ and $\ell_2$ of $V_\rho$ satisfying
	\[\ell_i\left(\rho\left(\begin{bmatrix}
		x & 0 \\
		0 & 1
	\end{bmatrix}\right)v\right)=\omega(x)^{-1}\ell_i(v)\qquad\text{for every }x\in K^\times\text{ and }v\in V_\rho\]
	are linearly dependent.
\end{lemma}
\begin{proof}
	Note $\op{Res}^B_AV_\pi$ is isomorphic to the space of all functions $\varphi\colon K^\times\to\CC$, and $A$ acts on this space by the formula
	\[\left(\rho\left(\begin{bmatrix}
		\alpha & 0 \\
		0 & 1
	\end{bmatrix}\right)\varphi\right)(x)=\varphi(x\alpha).\]
	Indeed, we may an $f\in V_\pi$ to the function $\varphi$ defined by
	\[\varphi(x)\coloneqq f\left(\begin{bmatrix}
		x & 0 \\
		0 & 1
	\end{bmatrix}\right).\]
	On the other hand, the identity
	\[\begin{bmatrix}
		1 & \beta \\
		0 & 1
	\end{bmatrix}\begin{bmatrix}
		\alpha & 0 \\
		0 & 1
	\end{bmatrix}=\begin{bmatrix}
		\alpha & \beta \\
		0 & 1
	\end{bmatrix}\]
	implies that $\varphi$ also determines $f$.

	There exists now exactly one nonzero function $\varphi$ (up to a multiplication by a scalar) such that $\varphi(x\alpha)=\omega(\alpha)^{-1}\varphi(x)$. This function is defined by $\varphi(x)=\omega(x)^{-1}\varphi(1)$. Using \eqref{eq:decomp-res-rho} and the assumption that the inverse $\omega$ is non-exceptional, this means that $\omega$ is an eigenvalue operating on $V_\rho$ and the subspace of eigenvectors of $A$ belonging to $\omega$ is one-dimensional.

	Now let $\zeta$ be a generator of the cyclic group $K^\times$ and define a linear map $T\colon V_\rho\to V_\rho$ by
	\[Tv=\rho\left(\begin{bmatrix}
		\zeta & 0 \\
		0 & 1
	\end{bmatrix}\right)v-\omega(\zeta)^{-1}v.\]
	Then
	\[\ker T=\left\{v\in V_\rho:\rho\left(\begin{bmatrix}
		x & 0 \\
		0 & 1
	\end{bmatrix}\right)v=\omega(x)^{-1}v\text{ for all }x\in K^\times\right\}\]
	is the subspace of eigenvectors belonging to $\omega^{-1}$. We proved that $\dim\ker T=1$. Hence, $\dim T(V_\rho)=\dim\rho-1$. However, $T(V)\subseteq\ker\ell_i$, and $\dim\ker\ell_i=\dim\rho-1$. Hence, $\ker\ell_1=T(V)=\ker\ell_2$. It follows that $\ell_1$ and $\ell_2$ are linearly dependent.
\end{proof}
\begin{theorem} \label{thm:def-gamma}
	Let $\rho$ be a higher-dimensional irreducible representation of $G$, and let $\omega$ be a character of $K^\times$ which is not exceptional for $\rho$. Then there exists a complex number $\Gamma_\rho(\omega)$ such that for every Whittaker function $W_v$ $W_v$ of $\rho$, we have
	\[\Gamma_\rho(\omega)\sum_{x\in K^\times}W_v\left(\begin{bmatrix}
		x & 0 \\
		0 & 1
	\end{bmatrix}\right)\omega(x)=\sum_{x\in K^\times}W_v\left(\begin{bmatrix}
		0 & 1 \\
		x & 0
	\end{bmatrix}\right)\omega(x).\]
\end{theorem}
\begin{proof}
	Define linear functionals $\ell_i$, $i=1,2$ of $V_\rho$ by
	\[\ell_1(v)\coloneqq\sum_{x\in K^\times}W_v\left(\begin{bmatrix}
		x & 0 \\
		0 & 1
	\end{bmatrix}\right)\qquad\text{and}\qquad\ell_2(v)\coloneqq\sum_{x\in K^\times}W_v\left(\begin{bmatrix}
		0 & 1 \\
		x & 0
	\end{bmatrix}\right)\omega(x).\]
	Then
	\[\ell_i\left(\rho\left(\begin{bmatrix}
		\alpha & 0 \\
		0 & 1
	\end{bmatrix}\right)v\right)=\omega(\alpha)^{-1}\ell_i(v)\qquad\text{for }i=1,2\]
	for every $\alpha\in K^\times$ and every $v\in V_\rho$. For example,
	\begin{align*}
		\ell_2\left(\rho\left(\begin{bmatrix}
			\alpha & 0 \\
			0 & 1
		\end{bmatrix}\right)v\right) &= \sum_{x\in K^\times}W_{\rho\left(\begin{bmatrix}
			\alpha & 0 \\
			0 & 1
		\end{bmatrix}\right)v}\left(\begin{bmatrix}
			0 & 1 \\
			x & 0
		\end{bmatrix}\right)\omega(x) \\
		&= \sum_{x\in K^\times}W_v\left(\begin{bmatrix}
			0 & 1 \\
			x\alpha & 0
		\end{bmatrix}\right)\omega(x)=\omega(\alpha)^{-1}\sum_{y\in K^\times}W_v\left(\begin{bmatrix}
			0 & 1 \\
			y & 0
		\end{bmatrix}\right)\omega(y) \\
		&= \omega(\alpha)^{-1}\ell_2(y).
	\end{align*}
	It follows from \Cref{lem:almost-gamma-func} that $\ell_2$ is a multiple of $\ell_1$ by a constant.\footnote{Notably, $\ell_2$ is nonzero (and hence $\ell_1$ is nonzero): the proof of \Cref{lem:almost-gamma-func} gives $v\in V_\pi$ which is an eigenvector of $A$ with eigenvalue $\omega^{-1}$. Then we can compute $\ell_2(v)=(q-1)W_v(1)$, so $\ell_2(v)=0$ implies $W_v(1)=0$. However, this means $W_v(a)=W_{av}(1)=\omega^{-1}(a)W_v(1)=0$ for any $a\in A$, so \Cref{lem:ker-r-computation} below implies that $v\in J(V_\rho)$, so $v=0$ because $v\in V_\pi$, which is a contradiction.} We denote this constant by $\Gamma_\rho(\omega)$.
\end{proof}

The complex-valued function $\Gamma_\rho(\omega)$ defined for every non-exceptional character $\omega$ of $K^\times$ will play an important role in the computation of the character table of $G$.

\section{Determination of \texorpdfstring{$\rho$}{ rho} by \texorpdfstring{$\Gamma_\rho$}{ Gamma rho}}
Let $\rho$ be a higher-dimensional representation of $G$. For every $v\in V_\rho$, let $W_v$ be the corresponding Whittaker function of $\rho$, and let $r$ be the homomorphism of $V_\rho$ into the space $F\left(K^\times,\CC\right)$ of all functions $\varphi\colon K^\times\to\CC$ defined by $r(v)\coloneqq\op{Res}_A^GW_v$. If we define an operation of $K^\times$ on $F\left(K^\times,\CC\right)$ by $(\alpha\cdot\varphi)(x)=\varphi(x\alpha)$ and identify $A$ with $K^\times$, then $r$ is also an $A$-homomorphism.
\begin{lemma} \label{lem:ker-r-computation}
	The homomorphism $r$ is surjective, and $\ker r=J(V_\rho)$.
\end{lemma}
\begin{proof}
	We start by determining the kernel of $r$. Let $v\in J(V_\rho)$; then for every $\alpha\in K^\times$, we choose a $\beta\in K$ such that $\psi(\alpha\beta)\ne0$. Then
	\begin{align*}
		W_v\left(\begin{bmatrix}
			\alpha & 0 \\
			0 & 1
		\end{bmatrix}\right) &= W_{\rho\left(\begin{bmatrix}
			1 & \beta \\
			0 & 1
		\end{bmatrix}\right)v}\left(\begin{bmatrix}
			\alpha & 0 \\
			0 & 1
		\end{bmatrix}\right) \\
		&= W_v\left(\begin{bmatrix}
			\alpha & \alpha\beta \\
			0 & 1
		\end{bmatrix}\right) \\
		&= W_v\left(\begin{bmatrix}
			1 & \alpha\beta \\
			0 & 1
		\end{bmatrix}\begin{bmatrix}
			\alpha & 0 \\
			0 & 1
		\end{bmatrix}\right) \\
		&= \psi(\alpha\beta)W_v\left(\begin{bmatrix}
			\alpha & 0 \\
			0 & 1
		\end{bmatrix}\right).
	\end{align*}
	Hence, $W_v\left(\begin{bmatrix}
		\alpha & 0 \\
		0 & 1
	\end{bmatrix}\right)=0$; i.e., $v\in\ker r$.

	To prove that $\ker r\subseteq J(V_\rho)$, recall again the $P$-decomposition $V=J(V_\rho)\oplus V_\pi$ (see \Cref{lem:ind-psi-has-reps}). Then $V_\pi\cap\ker r$ is left-invariant by $P$. This follows from the decomposition $P=AU$ and from the following two computations: let $v\in V_\pi\cap\ker r$; then
	\begin{align}
		W_{\rho\left(\begin{bmatrix}
			\alpha & 0 \\
			0 & 1
		\end{bmatrix}\right)v}\left(\begin{bmatrix}
			\alpha & 0 \\
			0 & 1
		\end{bmatrix}\right)&=W_v\left(\begin{bmatrix}
			\alpha\alpha' & 0 \\
			0 & 1
		\end{bmatrix}\right)=0, \notag \\
		W_{\rho\left(\begin{bmatrix}
			1 & \beta \\
			0 & 1
		\end{bmatrix}\right)v}\left(\begin{bmatrix}
			\alpha & 1 \\
			0 & 1
		\end{bmatrix}\right)&=\psi(\alpha\beta)W_v\left(\begin{bmatrix}
			\alpha & 0 \\
			0 & 1
		\end{bmatrix}\right)=0. \label{eq:whittaker-apply-u}
	\end{align}
	It follows that $V_\pi\cap\ker r=0$ or $V_\pi\cap\ker r=V_\pi$ because $V_\pi$ is $P$-irreducible.

	Assume that $V_\pi\cap\ker r=V_\pi$. Then by the first part of the prof, $W_v(a)=0$ for every $v\in V$ and every $a\in A$. Hence, if $g\in G$, then $W_v(G)=W_{\rho(g)v}(1)=0$, i.e., $v=0$, which is a contradiction. It follows that $V_\pi\cap\ker r=0$; hence $\ker r=J(V_\rho)$. This fact implies that $\dim\im r=\dim V_\rho-\dim J(V_\rho)=q-1=\dim F\left(K^\times,\CC\right)$. Hence, $\im r=F\left(K^\times,\CC\right)$.
\end{proof}

The center $Z$ of $G$ consists of the scalar matrices and is therefore canonically isomorphic to $K^\times$. The restriction of $\rho$ to $Z$ can therefore be identified with a character $\omega_P$ of $K^\times$, called the central character of $\rho$.
\begin{proposition}
	A cuspidal representation $\rho$ of $G$ is uniquely determined by its $\Gamma$-function and its central character.
\end{proposition}
\begin{proof}
	Let $\rho$ and $\rho'$ be two cuspidal representation of $G$. Then by \Cref{prop:cuspidal-dim}, $\op{Res}^G_P\rho=\pi=\op{Res}^G_P\rho'$. If $\rho$ and $\rho'$ coincide on $Z$, then they coincide on $B$ since $B=ZP$. Suppose in addition that $\Gamma_\rho=\Gamma_{\rho'}$. We have to prove that $\rho=\rho'$. The Bruhat decomposition $G=B\sqcup BwU$ implies that it suffices to show that $\rho(w)=\rho'(w)$.

	Both representations $\rho$ and $\rho'$ are of dimension $q-1$. We can therefore assume that both of them act on the space $V$. For every $v\in V$, let $W_v$ and $W'_v$ be Whittaker functions of $\rho$ and $\rho'$, respectively. We know that $J(V_\rho)=J(V_{\rho'})=0$; hence, by \Cref{lem:ker-r-computation} the maps $v\mapsto\op{Res}^G_AW_v$ and $v\mapsto\op{Res}^G_AW'_v$ are $A$-isomorphisms of $V$ onto $F\left(K^\times,\CC\right)$. Hence, without loss of generality, we can assume that $\op{Res}^G_AW_v=\op{Res}_A^GW'_v$.\footnote{I find this argument unsatisfactory. Here's an alternate: fix $v_0\in V$ which is an $A$-eigenvector with eigenvalue $\omega$, which implies $W_{v_0}(1),W_{v_0}'(1)\ne0$ as explained previously. Scaling, we may assume $W_{v_0}(1)=W_{v_0}(1)$. Note $W_{v_0}(a)=W'_{v_0}(a)$ for any $a\in A$. Properties of Whittaker functions allow us to extend this to $W_{auv_0}(a')=\psi\left((aa')u(aa')^{-1}\right)\omega(aa')W_{v_0}(1)=W'_{auv_0}(a')$ for any $a,a'\in A$ and $u\in U$. Thus, $\op{Res}^G_AW_v=\op{Res}^G_AW_v'$ for any $v\in Pv_0$, but $Pv_0$ spans $V$ because $\op{Res}^G_PV=V_\pi$ is $P$-irreducible.}

	Therefore, by \Cref{thm:def-gamma}, the assumption $\Gamma_{\rho}=\Gamma_{\rho'}$ implies that for every character $\omega$ of $K^\times$,
	\[\sum_{x\in K^\times}W_v\left(\begin{bmatrix}
		0 & 1 \\
		x & 0
	\end{bmatrix}\right)\omega(x)=\sum_{x\in K^\times}W'_v\left(\begin{bmatrix}
		0 & 1 \\
		x & 0
	\end{bmatrix}\right)\omega(x).\]
	This implies by Artin's lemma that
	\[W_v\left(\begin{bmatrix}
		0 & 1 \\
		x & 0
	\end{bmatrix}\right)=W'_v\left(\begin{bmatrix}
		0 & 1 \\
		x & 0
	\end{bmatrix}\right)\qquad\text{for every }x\in K^\times.\]
	Hence,
	\begin{align*}
		W_{\rho(w)v}\left(\begin{bmatrix}
			x & 0 \\
			0 & 1
		\end{bmatrix}\right) &= W_v\left(\begin{bmatrix}
			x & 0 \\
			0 & 1
		\end{bmatrix}\begin{bmatrix}
			0 & 1 \\
			1 & 0
		\end{bmatrix}\right) \\
		&= W_v\left(\begin{bmatrix}
			0 & 1 \\
			x^{-1} & 0
		\end{bmatrix}\begin{bmatrix}
			x & 0 \\
			0 & x
		\end{bmatrix}\right) \\
		&= W_{\rho\left(\begin{bmatrix}
			x & 0 \\
			0 & x
		\end{bmatrix}\right)v}\left(\begin{bmatrix}
			0 & 1 \\
			x^{-1} & 0
		\end{bmatrix}\right) \\
		&= W_{\rho'\left(\begin{bmatrix}
			x & 0 \\
			0 & x
		\end{bmatrix}\right)v}\left(\begin{bmatrix}
			0 & 1 \\
			x^{-1} & 0
		\end{bmatrix}\right) \\
		&= W'_{\rho'\left(\begin{bmatrix}
			x & 0 \\
			0 & x
		\end{bmatrix}\right)v}\left(\begin{bmatrix}
			0 & 1 \\
			x^{-1} & 0
		\end{bmatrix}\right) \\
		&= W'_{\rho'(w)}\left(\begin{bmatrix}
			x & 0 \\
			0 & 1
		\end{bmatrix}\right).
	\end{align*}
	Hence, by \Cref{lem:ker-r-computation}, $W_{\rho(w)v}=W'_{\rho'(w)v}$, and hence $\rho(w)=\rho'(w)$.
\end{proof}

\section{The Bessel function of a representation}
Let $\rho$ be a higher-dimensional representation of $G$. Then $\dim V_\rho>2\ge\dim J(V_\rho)$ except in the case where $q=3$ and $\dim\rho=2$. In this case, $\rho$ is however cuspidal and $J(V_\rho)=0$. Therefore, $J(V_\rho)\ne V_\rho$ in all cases.

As $U$ is an abelian group, $\op{Res}^G_U\rho$ decomposes into a direct sum of characters. One of them must be different from the unit character. Indeed, otherwise we would have that
\[\rho\left(\begin{bmatrix}
	1 & \beta \\
	0 & 1
\end{bmatrix}\right)v=v\]
for every $\beta\in K$ and every $v\in V_\rho$. Then by \eqref{eq:whittaker-apply-u},
\[W_v\left(\begin{bmatrix}
	\alpha & 0 \\
	0 & 1
\end{bmatrix}\right)=\psi(\alpha\beta)W_v\left(\begin{bmatrix}
	\alpha & 0 \\
	0 & 1
\end{bmatrix}\right)\]
for every $\alpha\in K^\times$. Hence, $\op{Res}^G_AW_v=0$, and hence $v\in J(V_\rho)$ by \Cref{lem:ker-r-computation}. This contradicts the inequality $J(V_\rho)\ne V_\rho$. There exists therefore a non-unit character $\psi_1$ of $K^+$ and a nonzero vector $v_1'\in V_\rho$ such that
\[\rho\left(\begin{bmatrix}
	1 & \beta \\
	0 & 1
\end{bmatrix}\right)v_1'=\psi_1(\beta)v_1'\qquad\text{for every }\beta\in K.\]
By \cref{sec:reps-of-p}, there exists an $\alpha\in K^\times$ such that $\psi_1(\beta)=\psi(\alpha\beta)$. Using the identity
\[\begin{bmatrix}
	\alpha & 0 \\
	0 & 1
\end{bmatrix}\begin{bmatrix}
	1 & \alpha^{-1}y \\
	0 & 1
\end{bmatrix}\begin{bmatrix}
	\alpha^{-1} & 0 \\
	0 & 1
\end{bmatrix}=\begin{bmatrix}
	1 & y \\
	0 & 1
\end{bmatrix}\]
and replacing $v_1'$ by $v_1\coloneqq\rho\left(\begin{bmatrix}
	\alpha & 0 \\
	0 & 1
\end{bmatrix}\right)v_1'$, we get that
\begin{equation}
	\rho\left(\begin{bmatrix}
		1 & \beta \\
		0 & 1
	\end{bmatrix}\right)v_1=\psi(\beta)v_1\qquad\text{for every }\beta\in K. \label{eq:v1-is-psi-eigen}
\end{equation}
It follows that if $\alpha\in K^\times$, then
\[\psi(\beta)W_{v_1}\left(\begin{bmatrix}
	\alpha & 0 \\
	0 & 1
\end{bmatrix}\right)=\psi(\alpha\beta)W_{v_1}\left(\begin{bmatrix}
	\alpha & 0 \\
	0 & 1
\end{bmatrix}\right).\]
If $\alpha\ne1$, then we may conclude that
\[W_{v_1}\left(\begin{bmatrix}
	\alpha & 0 \\
	0 & 1
\end{bmatrix}\right)=0.\]
Further, \eqref{eq:v1-is-psi-eigen} implies that $v_1\notin J(V_\rho)$. Hence, $\op{Res}_AW_{v_1}\ne0$ by \Cref{lem:ker-r-computation} and therefore $W_{v_1}(1)\ne0$. The vector $v_1$ is said to be a \textit{Bessel vector} for $\rho$.

If $v_2$ if an additional Bessel vector for $\rho$, then by the last paragraph, there exists a $\zeta\in\CC$ such that $W_{v_1}(a)=\zeta W_{v_2}(a)$ for every $a\in A$. Using \Cref{lem:ker-r-computation} once again, we conclude that $v_1-\zeta v_2\in J(V_\rho)$. Hence,
\[\psi(\beta)(v_1-\zeta v_2)=\rho\left(\begin{bmatrix}
	1 & \beta \\
	0 & 1
\end{bmatrix}\right)(v_1-\zeta v_2)=v_1-\zeta v_2\qquad\text{for every }\beta\in K.\]
Hence, $v_1=\zeta v_2$. Thus, up to a scalar multiple, there exists only one Bessel vector for $\rho$. We use this vector to define the \textit{Bessel function} $J_\rho\colon G\to\CC$ of $\rho$ by
\[J_\rho(g)\coloneqq\left(W_{v_1}\left(\begin{bmatrix}
	1 & 0 \\
	0 & 1
\end{bmatrix}\right)\right)^{-1}W_{v_1}(g).\]
Clearly, $J_\rho(g)$ does not depend on the particular Bessel vector $v_1$ which is used in its definition. Note that $J_\rho$ is also a Whittaker function for $\rho$. Therefore,
\[J_\rho(gu)=J_\rho(ug)=\psi(u)J_\rho(g)\qquad\text{for }u\in U\text{ and }g\in G.\]
Also,
\begin{equation}
	J_\rho(1)=1\qquad\text{and}\qquad J_\rho(a)=0\text{ if }a\ne1\text{ and }a\in A. \label{eq:bessel-on-a}
\end{equation}
Therefore, if a character $\omega$ of $K^\times$ is not exceptional for $\rho$, we have by \Cref{thm:def-gamma} that
\begin{equation}
	\Gamma_\rho(\omega)=\sum_{x\in K^\times}J_\rho\left(\begin{bmatrix}
		0 & 1 \\
		x & 0
	\end{bmatrix}\right)\omega(x). \label{eq:general-gamma}
\end{equation}
One can use this formula to define $\Gamma_\rho(\omega)$ also for the exceptional characters. We shall use this formula in the next two sections in order to compute $\Gamma_\rho$.

\section{A computation of \texorpdfstring{$\Gamma_\rho(\omega)$}{ Gamma rho(omega)} for a non-cuspidal \texorpdfstring{$\rho$}{ rho}}
Let $\rho$ be a higher-dimensional irreducible representation of $G$ which is not cuspidal. Then $\rho$ is a component of $\widehat\mu\coloneqq\op{Ind}^G_B\mu$, where $\mu$ is a character of $B$ that corresponds to the pair of characters $(\mu_1,\mu_2)$ of $K^\times$. We may consider therefore as an irreducible $G$-subspcae of $\op{Ind}^G_BV_\mu$. Every element of $V_\rho$ appears then as a function $f\colon G\to\CC$ such that
\[f(bg)=\mu(b)f(g)\qquad\text{for }b\in B,g\in G.\]
The action of $G$ on $V_\rho$ is given by $(\rho(s)f)(g)=f(gs)$.

We shall use this description of $V_\rho$ in order to give a concrete Whittaker model for $\rho$. For every $f\in V_\rho$, let $W_f\colon G\to\CC$ be the function defined by
\begin{equation}
	W_f(g)\coloneqq\sum_{z\in K}f\left(w\begin{bmatrix}
		1 & z \\
		0 & 1
	\end{bmatrix}g\right)\psi(z)^{-1}. \label{eq:non-cusp-whittaker-model}
\end{equation}
It is easy to check that $W_f$ satisfies the equalities characterizing Whittaker's functions
\[\begin{array}{cc}
	W_v=0\iff v=0, \\
	W_{c_1v_1+c_2v_2}=c_1W_{v_1}+c_2W_{v_2} & \text{for }c_1,c_2\in\CC, \\
	W_v(ug)=\psi(u)W_v(g) & \text{for }u\in U\text{ and }g\in G, \\
	W_{\rho(s)v}=W_v(gs) & \text{for }s,g\in G.
\end{array}\]
The map $f\mapsto W_f$ is therefore a linear map from $V_\rho$ into $\op{ind}^G_UV_\psi$. We show that its kernel is zero by constructing a specific function $f_1\in V_\rho$ such that $W_{f_1}\ne0$. This implies that the map is injective and hence that $\{W_f:f\in V_\rho\}$ is a Whittaker model for $\rho$.

Using Bruhat's decomposition $G=B\sqcup BwU$, we define $f_1$ by
\begin{align}
	f_1(b) \coloneqq{}& 0, \label{eq:f1-on-b} \\
	f_1(bwu) \coloneqq{}& \mu(b)\psi(u)\qquad\text{for }b\in B\text{ and }u\in U. \label{eq:f1-on-bwu}
\end{align}
Then $f_1$ is a nonzero element of $\op{Ind}^G_BV_\mu$, and it satisfies
\begin{equation}
	f_1(gu) = \psi(u)f_1(g)\qquad\text{for }g\in G\text{ and }u\in U. \label{eq:f1-eigenvalue}
\end{equation}
We prove that $f_1$ belongs to $V_\rho$.

If $\dim\rho=q+1$, then $V_\rho=\op{Ind}^G_BV_\mu$, and there is nothing to prove. We can therefore assume that $\dim\rho=q$. In this case, $\widehat\mu=\rho'\oplus\mu$, where $\rho'$ is a one-dimensional representation of $G$; hence, $\op{Ind}^G_BV_\mu=V_{\rho'}\oplus V_\rho$. We can therefore write $f_1=f'+f$, where $f'\in V_{\rho'}$ and $f\in V_\rho$. Now, for a fixed $u\in U$, the function $g\mapsto f'(gu)$ and $g\mapsto\psi(u)f'9g)$ belong to $V_{\rho'}$, whereas the functions $g\mapsto f(gu)$ and $g\mapsto\psi(u)f(g)$ belong to $V_\rho$. Using \eqref{eq:f1-eigenvalue}, we conclude that
\begin{equation}
	f'(gu)=\psi(u)f'(g)\qquad\text{for }g\in G. \label{eq:fp-computation1}
\end{equation}
By the proof of \Cref{lem:to-reducible}, the function $f'$ satisfies
\begin{equation}
	f'(g)=f'(1)\mu_1(\det g)\qquad\text{for }g\in G. \label{eq:fp-computation2}
\end{equation}
It follows from \eqref{eq:fp-computation1} that $f'(1)=\psi(u)f'(1)$ for every $u\in U$. Hence, $f'(1)=0$. This implies $f'=0$ by \eqref{eq:fp-computation2}. It follows that $f_1=f\in V_\rho$. Computing $W_{f_1}$ for $u'\in U$, we find
\[W_{f_1}(u')=\sum_{u\in U}f_1(wuu')\psi(u)^{-1}=\sum_{u\in U}\psi(uu')\psi(u)^{-1}=q\psi(u').\]
Hence, $W_{f_1}\ne0$, and our contention is proved.

Note that \eqref{eq:f1-eigenvalue} implies now that $f_1$ is a Bessel vector for $\rho$. We shall use this information in order to compute the Bessel function $J_\rho$, but first let us sum up the results proved in this section up to now.
\begin{lemma}
	Let $\rho$ be a non-cuspidal higher-dimensional irreducible representation of $G$. Then the subspace $\{W_f:f\in V_\rho\}$ of $\op{Ind}^G_UV_\psi$ defined by \eqref{eq:non-cusp-whittaker-model} is a Whittaker model for $\rho$, and the function $f_1$ defined by \eqref{eq:f1-on-b} and \eqref{eq:f1-on-bwu} is a Bessel vector for $\rho$ in this model.
\end{lemma}
In order to compute $\Gamma_\rho(\omega)$, we now have to compute by \eqref{eq:general-gamma}
\begin{equation}
	W_{f_1}\left(\begin{bmatrix}
		0 & 1 \\
		x & 0
	\end{bmatrix}\right)=\sum_{z\in K}f_1\left(\begin{bmatrix}
		0 & 1 \\
		1 & 0
	\end{bmatrix}\begin{bmatrix}
		1 & z \\
		0 & 1
	\end{bmatrix}\begin{bmatrix}
		0 & 1 \\
		x & 0
	\end{bmatrix}\right)\psi(z)^{-1}. \label{eq:compute-w-f1}
\end{equation}
For $z=0$, we have by \eqref{eq:f1-on-b} that
\[f_1\left(\begin{bmatrix}
	0 & 1 \\
	1 & 0
\end{bmatrix}\begin{bmatrix}
	0 & 1 \\
	x & 0
\end{bmatrix}\right)=f_1\left(\begin{bmatrix}
	x & 0 \\
	0 & 1
\end{bmatrix}\right)=0.\]
If $z\ne0$, then
\[\begin{bmatrix}
	0 & 1 \\
	1 & 0
\end{bmatrix}\begin{bmatrix}
	1 & z \\
	0 & 1
\end{bmatrix}\begin{bmatrix}
	0 & 1 \\
	x & 0
\end{bmatrix}=\begin{bmatrix}
	-z^{-1} & x \\
	0 & zx
\end{bmatrix}\begin{bmatrix}
	0 & 1 \\
	1 & 0
\end{bmatrix}\begin{bmatrix}
	1 & (zx)^{-1} \\
	0 & 1
\end{bmatrix};\]
hence by \eqref{eq:f1-on-bwu},
\[f_1\left(\begin{bmatrix}
	0 & 1 \\
	1 & 0
\end{bmatrix}\begin{bmatrix}
	1 & z \\
	0 & 1
\end{bmatrix}\begin{bmatrix}
	0 & 1 \\
	x & 0
\end{bmatrix}\right)=\mu_1\left(-z^{-1}\right)\mu_2(zx)\psi\left((zx)^{-1}.\right)\]
It follows from \eqref{eq:compute-w-f1} that
\[W_{f_1}\left(\begin{bmatrix}
	0 & 1 \\
	x & 0
\end{bmatrix}\right)=\sum_{z\in K^\times}\mu_1\left(-z^{-1}\right)\mu_2(zx)\psi\left((zx)^{-1}-z\right)=\sum_{rs=-x^{-1}}\mu_1(r)^{-1}\mu_2(s)^{-1}\psi(s+r).\]
Also, by \eqref{eq:f1-on-bwu}
\[W_{f_1}\left(\begin{bmatrix}
	1 & 0 \\
	0 & 1
\end{bmatrix}\right)=\sum_{z\in K}f_1\left(w\begin{bmatrix}
	1 & z \\
	0 & 1
\end{bmatrix}\right)\psi(z)^{-1}=\sum_{z\in K}\psi(z)\psi(z)^{-1}=q.\]
Hence,
\[J_\rho\left(\begin{bmatrix}
	0 & 1 \\
	x & 0
\end{bmatrix}\right)=\frac1q\sum_{rs=-x^{-1}}\mu_1(r)^{-1}\mu_2(s)^{-1}\psi(s+r).\]
Substituting this value in formula \eqref{eq:general-gamma}, we obtain
\begin{align*}
	\Gamma_\rho(\omega) &= \sum_{x\in K^\times}J_\rho\left(\begin{bmatrix}
		0 & 1 \\
		x & 0
	\end{bmatrix}\right)\omega(x) \\
	&= \frac1q\sum_{x\in K^\times}\sum_{rs=-x^{-1}}\mu_1(r)^{-1}\mu_2(s)^{-1}\psi(r)\psi(s)\omega(x) \\
	&= \frac1q\frac1q\sum_{x\in K^\times}\sum_{rs=-x^{-1}}\mu_1(r)^{-1}\mu_2(s)^{-1}\psi(r)\psi(s)\omega\left(-r^{-1}s^{-1}\right) \\
	&= \frac{\omega(-1)}q\sum_{r\in K^\times}\mu_1(r)^{-1}\omega(r)^{-1}\psi(r)\sum_{s\in K^\times}\mu_2(s)^{-1}\omega(s)^{-1}\psi(s).
\end{align*}
Now, recall that for a character $\chi$ of $K^\times$, one defines the Gauss sum
\[G(\chi,\psi)\coloneqq\sum_{x\in K^\times}\chi(x)\psi(x).\]
We have therefore proved the following.
\begin{theorem} \label{thm:gamma-of-non-cusp}
	Let $\mu_1$ and $\mu_2$ be characters of $K^\times$, and let $\rho\coloneqq\rho_{(\mu_1,\mu_2)}$ be the corresponding irreducible representations of $G$. If $\omega$ is a character of $K^\times$, then
	\[\Gamma_\rho(\omega)=\frac{\omega(-1)}qG\left(\mu_1^{-1}\omega^{-1},\psi\right)G\left(\mu_2^{-1}\omega^{-1},\psi\right).\]
\end{theorem}
\begin{remark}
	It is well-known that $\left|G(\chi,\psi)\right|=\sqrt q$ for every character $\chi$ of $K^\times$; hence, $\left|\Gamma_\rho(\omega)\right|=1$.
\end{remark}
\begin{remark}
	If $\psi'$ is another non-unit character of $K^+$, then there exists an $\alpha$ such that $\psi'(x)=\psi(\alpha x)$. It follows that $G(\chi,\psi')=\chi(\alpha)^{-1}G(\chi,\psi)$. Hence, if we denote by $\Gamma_\rho'$ the $\Gamma$-function of $\rho$ obtained by using $\psi'$ instead of $\psi$, we obtain $\Gamma_\rho'(\omega)=\omega(\alpha)^2\mu_1(\alpha)\mu_2(\alpha)\Gamma_\rho(\omega)$.
\end{remark}

\section{A computation of \texorpdfstring{$\Gamma_\rho(\omega)$}{ Gamma rho(omega)} for a cuspidal \texorpdfstring{$\rho$}{ rho}}
Let $\nu$ be a non-decomposable character of $L^\times$, and let $\rho\coloneqq\rho_\nu$ be the corresponding cuspidal representation of $G$. Then $\rho(w')$ acts on a function $f\colon K^\times\to\CC$ by
\begin{equation}
	(\rho(w')f)(x)=\sum_{\substack{y\in K^\times\\\op Nt=u}}\nu(y)^{-1}j(yx)f(y), \label{eq:cuspidal-wp}
\end{equation}
where $j=j_\nu$ is the function of $K^\times$ given by
\[j(u)\coloneqq -q^{-1}\sum_{\substack{t\in L\\\op Nt=u}}\psi\left(t+\overline t\right)\nu(t)\]
(see \cref{sec:def-cusp}). In particular, we apply \eqref{eq:cuspidal-wp} to the function
\[f'(x)\coloneqq J_\rho\left(\begin{bmatrix}
	x & 0 \\
	0 & 1
\end{bmatrix}\right)\]
which is equal to $1$ for $x=1$ and equal to $0$ if $x\ne1$ (by \eqref{eq:bessel-on-a}). Also,
\[(\rho(w')f')(x)=J_\rho\left(\begin{bmatrix}
	x & 0 \\
	0 & 1
\end{bmatrix}\begin{bmatrix}
	0 & 1 \\
	-1 & 0
\end{bmatrix}\right)=J_\rho\left(\begin{bmatrix}
	0 & x \\
	-1 & 0
\end{bmatrix}\right).\]
Hence, \eqref{eq:cuspidal-wp} implies
\[J_\rho\left(\begin{bmatrix}
	0 & x \\
	-1 & 0
\end{bmatrix}\right)=j(x).\]
Also,
\[\begin{bmatrix}
	0 & 1 \\
	x & 0
\end{bmatrix}=\begin{bmatrix}
	0 & -x^{-1} \\
	-1 & 0
\end{bmatrix}\begin{bmatrix}
	-x & 0 \\
	0 & -x
\end{bmatrix};\]
hence by \eqref{eq:def-cusp-b},
\[J_\rho\left(\begin{bmatrix}
	0 & 1 \\
	x & 0
\end{bmatrix}\right)=\left(\rho\left(\begin{bmatrix}
	-x & 0 \\
	0 & -x
\end{bmatrix}\right)J_\rho\right)\left(\begin{bmatrix}
	0 & -x^{-1} \\
	-1 & 0
\end{bmatrix}\right)=\nu(-x)J_\rho\left(\begin{bmatrix}
	0 & -x^{-1} \\
	-1 & 0
\end{bmatrix}\right).\]
Hence, if $\omega$ is a character of $K^\times$, then
\begin{align*}
	\Gamma_\rho(\omega) &= \sum_{x\in K^\times}J_\rho\left(\begin{bmatrix}
		0 & 1 \\
		x & 0
	\end{bmatrix}\right)\omega(x) \\
	&= \sum_{x\in K^\times}\nu(-x)j\left(-x^{-1}\right)\omega(x) \\
	&= -q^{-1}\sum_{x\in K^\times}\nu(-x)\omega(x)\sum_{\substack{t\in L^\times\\\op Nt=-x^{-1}}}\psi\left(t+\overline t\right)\nu(t) \\
	&= -q^{-1}\omega(-1)\sum_{t\in L^\times}\nu\left(\overline t\right)^{-1}\omega\left(t\overline t\right)^{-1}\psi\left(t+\overline t\right) \\
	&= -q^{-1}\omega(-1)\sum_{t\in L^\times}\nu\left(t\right)^{-1}\omega\left(t\overline t\right)^{-1}\psi\left(t+\overline t\right).
\end{align*}
Note that the last sum is a Gauss sum for the field $L$. Hence, we have proved a result similar to \Cref{thm:gamma-of-non-cusp}.
\begin{theorem}
	Let $\nu$ be a non-decomposable character of $L^\times$ and let $\rho\coloneqq\rho_\nu$ be the corresponding cuspidal representation of $G$. Then
	\[\Gamma_\rho(\omega)=-\frac{\omega(-1)}q\sum_{t\in L^\times}\nu(t)^{-1}\omega\left(t\overline t\right)^{-1}\psi\left(t+\overline t\right)=-\frac{\omega(-1)}qG_L\left(\nu^{-1}\cdot\left(\omega\circ\op N_{L/K}\right)^{-1},\psi\circ\op{Tr}_{L/K}\right)\]
	for every character $\omega$ of $K$.\todo{Check that this is right.}
\end{theorem}
\begin{remark}
	As in the non-cuspidal case, $\left|\Gamma_\rho(\omega)\right|=1$ because $\left|L\right|=q^2$. Also, if $\psi'(x)=\psi(\alpha x)$ is another character of $K^\times$, then $\Gamma'_\rho(\omega)=\nu(\alpha)\omega(\alpha)^2\Gamma_\rho(\omega)$.
\end{remark}

\section{The characters of \texorpdfstring{$G$}{ G}}
We conclude our exposition on the representations of $G$ with a computation of its characters table, i.e., the values of the irreducible higher-dimensional characters at the representatives of the conjugacy classes of $G$. We recall that there are four families of representatives.
\begin{align*}
	c_1(x) &= \begin{bmatrix}
		x & 0 \\
		0 & x
	\end{bmatrix} \qquad\text{for }x\in K^\times, \\
	c_2(x) &= \begin{bmatrix}
		x & 1 \\
		0 & x
	\end{bmatrix} \qquad\text{for }x\in K^\times, \\
	c_3(x,y) &= \begin{bmatrix}
		x & 0 \\
		0 & y
	\end{bmatrix} \qquad\text{for }x,y\in K^\times\text{ and }x\ne y, \\
	c_4(z) &= \begin{bmatrix}
		0 & -z\overline z \\
		1 & z+\overline z
	\end{bmatrix} \qquad\text{for }z\in L\setminus K.
\end{align*}

\subsection{The Gelfand--Graev Character}
We start with the computation of the Gelfand--Graev character: let $\widehat\psi\coloneqq\op{Ind}^G_U\psi$. Denote by $\widetilde\psi$ the function on $G$ that coincides with $\psi$ on $U$ and is equal to zero elsewhere. Then
\[\chi_{\widehat\psi}(g)=\sum_{s\in U\backslash G}\widetilde\psi\left(sgs^{-1}\right).\]
In particular, if $g$ is not conjugate to an element of $U$, then $\chi_{\widetilde\psi}(g)=0$. The only eigenvalue of the elements of $U$ is $1$. It follows that representatives on which $\chi_{\widehat\psi}$ might not vanish are $c_1(1)$ and $c_2(1)$. Clearly, $\chi_{\widetilde\psi}(c_1(1))=\chi_{\widehat\psi}(1)=\dim\widehat\psi=(q-1)q(q+1)$. In order to compute the value of $\chi_{\widehat\psi}$ at $c_2(1)$, we prove the following lemma.
\begin{lemma} \label{lem:conj-classes-of-t}
	For any $s\in G$,
	\[s\begin{bmatrix}
		1 & 1 \\
		0 & 1
	\end{bmatrix}s^{-1}\in U\]
	if and only if $s\in B$.
\end{lemma}
\begin{proof}
	Let $s\coloneqq\begin{bmatrix}
		\alpha & \beta \\
		\gamma & \delta
	\end{bmatrix}$ be an element of $G$, and let $s^{-1}\coloneqq\begin{bmatrix}
		\alpha' & \beta '\\
		\gamma' & \delta'
	\end{bmatrix}$. Then $s\in B$ if and only if $\gamma=0$ (which is equivalent to $\gamma'=0$). The lemma therefore follows from
	\begin{align*}
		\begin{bmatrix}
			\alpha & \beta \\
			\gamma & \delta
		\end{bmatrix}\begin{bmatrix}
			1 & 1 \\
			0 & 1
		\end{bmatrix}\begin{bmatrix}
			\alpha' & \beta' \\
			\gamma' & \delta'
		\end{bmatrix} &= \begin{bmatrix}
			\alpha & \beta \\
			\gamma & \delta
		\end{bmatrix}\left(\begin{bmatrix}
			1 & 0 \\
			0 & 1
		\end{bmatrix}+\begin{bmatrix}
			0 & 1 \\
			0 & 0
		\end{bmatrix}\right)\begin{bmatrix}
			\alpha' & \beta' \\
			\gamma' & \delta'
		\end{bmatrix} \\
		&= \begin{bmatrix}
			1 & 0 \\
			0 & 1
		\end{bmatrix}+\begin{bmatrix}
			\alpha\gamma' & \alpha\delta' \\
			\gamma\gamma' & \gamma\delta'
		\end{bmatrix} \\
		&= \begin{bmatrix}
			1+\alpha\gamma' & \alpha\delta' \\
			\gamma\gamma' & 1+\gamma\delta'
		\end{bmatrix},
	\end{align*}
	which is in $U$ if and only if $\gamma=\gamma'=0$.
\end{proof}
The lemma implies that
\begin{align*}
	\chi_{\widehat\psi}\left(\begin{bmatrix}
		1 & 1 \\
		0 & 1
	\end{bmatrix}\right) &= \sum_{s\in U\backslash B}\psi\left(s\begin{bmatrix}
		1 & 1 \\
		0 & 1
	\end{bmatrix}s^{-1}\right) \\
	&= \sum_{\alpha,\delta\in K^\times}\psi\left(\begin{bmatrix}
		\alpha & 0 \\
		0 & \delta
	\end{bmatrix}\begin{bmatrix}
		1 & 1 \\
		0 & 1
	\end{bmatrix}\begin{bmatrix}
		\alpha^{-1} & 0 \\
		0 & \delta^{-1}
	\end{bmatrix}\right) \\
	&= \sum_{\alpha,\delta\in K^\times}\psi\left(\begin{bmatrix}
		1 & \alpha\delta^{-1} \\
		0 & 1
	\end{bmatrix}\right) \\
	&= (q-1)\sum_{x\in K^\times}\psi(x) \\
	&= 1-q.
\end{align*}
We have therefore proved the following.
\begin{theorem}
	Let $\widehat\psi\coloneqq\op{Ind}^G_U\widehat\psi$. Then
	\begin{align*}
		\chi_{\widehat\psi}(c_1(1)) &= (q-1)q(q+1), \\
		\chi_{\widehat\psi}(c_1(x)) &= 0 \qquad\text{if }x\ne1, \\
		\chi_{\widehat\psi}(c_2(1)) &= q-1, \\
		\chi_{\widehat\psi}(c_2(x)) &= 0 \qquad\text{if }x\ne1, \\
		\chi_{\widehat\psi}(c_3(x,y)) &= 0, \\
		\chi_{\widehat\psi}(c_4(z)) &= 0.
	\end{align*}
\end{theorem}

\subsection{Non-Cuspidal Characters}
Let $\mu$ be a character of $B$ which is defined by the pair of characters $(\mu_1,\mu_2)$ of $K^\times$ through the formula
\[\mu\left(\begin{bmatrix}
	\alpha & \beta \\
	0 & \delta
\end{bmatrix}\right)\coloneqq\mu_1(\alpha)\mu_2(\alpha).\]
Let $\widehat\mu\coloneqq\op{Ind}^G_B\mu$, and we compute $\chi_{\widehat\mu}$. First,
\[\chi_{\widehat\mu}(c_1(x)) = \chi\left(\begin{bmatrix}
	x & 0 \\
	0 & x
\end{bmatrix}\right) = \sum_{s\in B\backslash G}\mu\left(s\begin{bmatrix}
	x & 0 \\
	0 & x
\end{bmatrix}s^{-1}\right)=[G:B]\mu\left(\begin{bmatrix}
	x & 0 \\
	0 & x
\end{bmatrix}\right)=(q+1)\mu_1(x)\mu_2(x).\]
Second, as in \Cref{lem:conj-classes-of-t}, one proves that
\[s\begin{bmatrix}
	x & 1 \\
	0 & x
\end{bmatrix}s^{-1}\in B\iff s\in B.\]
Hence,
\[\chi_{\widehat\mu}(c_2(x))=\mu\left(\begin{bmatrix}
	x & 1 \\
	0 & x
\end{bmatrix}\right)=\mu_1(x)\mu_2(x).\]
In order to compute the value of $\chi_{\widehat\mu}$ at $c_3(x,y)$, we note that a direct computation shows that
\begin{equation}
	\begin{bmatrix}
		\alpha & \beta \\
		\gamma & \delta
	\end{bmatrix}\begin{bmatrix}
		x & 0 \\
		0 & y
	\end{bmatrix}\begin{bmatrix}
		\alpha & \beta \\
		\gamma & \delta
	\end{bmatrix}^{-1}\in B\iff\gamma=0\text{ or }\delta=0. \label{eq:c3-in-b}
\end{equation}
Elements of $B\backslash G$ that satisfy the left-hand side of \eqref{eq:c3-in-b} are therefore $1$ and $w$. Indeed,
\[\begin{bmatrix}
	\alpha & \beta \\
	\gamma & 0
\end{bmatrix}=\begin{bmatrix}
	\beta & \alpha \\
	0 & \gamma
\end{bmatrix}\begin{bmatrix}
	0 & 1 \\
	1 & 0
\end{bmatrix}.\]
Hence,
\begin{align*}
	\chi_{\widehat\mu}(c_3(x,y)) &= \mu\left(\begin{bmatrix}
		x & 0 \\
		0 & y
	\end{bmatrix}\right)+\mu\left(\begin{bmatrix}
		0 & 1 \\
		1 & 0
	\end{bmatrix}\begin{bmatrix}
		x & 0 \\
		0 & y
	\end{bmatrix}\begin{bmatrix}
		0 & 1 \\
		1 & 0
	\end{bmatrix}\right) \\
	&= \mu\left(\begin{bmatrix}
		x & 0 \\
		0 & y
	\end{bmatrix}\right)+\mu\left(\begin{bmatrix}
		y & 0 \\
		0 & x
	\end{bmatrix}\right) \\
	&= \mu_1(x)\mu_2(y)+\mu_1(y)\mu_2(x).
\end{align*}
Finally, recall that the eigenvalues of the elements $c_4(z)$ do not belong to $K$. Hence, the $c_4(z)$ are not conjugate to elements of $B$ and therefore $\chi_{\widehat\mu}(c_4(z))=0$.

If $\mu_1\ne\mu_2$, then $\widehat\mu$ is an irreducible representation of $G$. Its character has therefore been computed.

If $\mu_1=\mu_2$, then $\widehat\mu=\rho'\oplus\rho$, where $\rho'$ is a one-dimensional character given by the formula $\rho'(g)=\mu_1(\det g)$ (by \Cref{lem:to-reducible}). Hence, $\rho'(c_1(x))=\mu_1(x)^2$, $\rho'(c_2(x))=\mu_1(x)^2$, $\rho'(c_3(x,y))=\mu_1(x)\mu_1(y)$, and $\rho'(c_4(z))=\mu_1(z\overline z)$. The values of $\chi_\rho$ can therefore be computed from the above data and from the formula $\chi_\rho(g)=\chi_{\widehat\mu}(g)-\mu_1(\det g)$.

\subsection{Cuspidal Characters}
The most difficult computation is presented by cuspidal representations. Let $\nu$ be a non-decomposable character of $L^\times$, and let $\rho\coloneqq\rho_\nu$ be the corresponding cuspidal representation. As $V_\rho$ we choose, as we did in \cref{sec:def-cusp}, the space $F\left(K^\times,\CC\right)$ of all functions $f\colon K^\times\to\CC$. Let $W_f$ be the Whittaker functions for $\rho$, and let $J_\rho$ be its Bessel function. Then
\begin{equation}
	W_f(g)=\sum_{y\in K^\times}W_f\left(\begin{bmatrix}
		y & 0 \\
		0 & 1
	\end{bmatrix}\right)J_\rho\left(g\begin{bmatrix}
		y^{-1} & 0 \\
		0 & 1
	\end{bmatrix}\right). \label{eq:whittaker-of-cusp}
\end{equation}
Indeed, both sides of \eqref{eq:whittaker-of-cusp} are Whittaker functions, and they coincide on $A$ because $J_\rho(a)=0$ if $1\ne a\in A$ and $J_\rho(1)=1$. Hence, by \Cref{lem:ker-r-computation}, they coincide on $g$s (recall that $J(V_\rho)=0$).\footnote{More precisely, we see that the summation (as a function of $g$) lives in the image of $V_\rho$ in $\op{Ind}^G_U\psi$, and \Cref{lem:ker-r-computation} says that $r\colon\op{Ind}^G_U\psi\to F\left(K^\times,\CC\right)$ is injective on the image of $V_\rho$.} The same lemma implies that for every function $\varphi\colon K^\times\to\CC$, there exists an $f\in V_\rho$ such that
\[W_f\left(\begin{bmatrix}
	x & 0 \\
	0 & 1
\end{bmatrix}\right)=\varphi(x)\qquad\text{for }x\in K^\times.\]
It follows by \eqref{eq:whittaker-of-cusp} that
\begin{equation}
	(\rho(g)\varphi)(x)\coloneqq W_{\rho(g)f}\left(\begin{bmatrix}
		x & 0 \\
		0 & 1
	\end{bmatrix}\right)=W_f\left(\begin{bmatrix}
		x & 0 \\
		0 & 1
	\end{bmatrix}g\right)=\sum_{t\in K^\times}\varphi(y)J_\rho\left(\begin{bmatrix}
		x & 0 \\
		0 & 1
	\end{bmatrix}g\begin{bmatrix}
		y^{-1} & 0 \\
		0 & 1
	\end{bmatrix}\right). \label{eq:def-g-phi}
\end{equation}

The space $F\left(K^\times,\CC\right)$ has a natural basis $\left\{\varphi_\alpha:\alpha\in K^\times\right\}$, where $\varphi_\alpha(\alpha)=1$ and $\varphi_\alpha(x)$ if $x\ne\alpha$. Substituting $\varphi=\varphi_\alpha$ in \eqref{eq:def-g-phi}, we have
\[(\rho(g)\varphi_\alpha)(x)=J_\rho\left(\begin{bmatrix}
	x & 0 \\
	0 & 1
\end{bmatrix}g\begin{bmatrix}
	\alpha^{-1} & 0 \\
	0 & 1
\end{bmatrix}\right).\]
The collection of all the elements on the right-hand side gives a $(q-1)\times(q-1)$ matrix representing $\rho(g)$. The trace of this matrix is
\begin{equation}
	\chi_\rho(g) = \sum_{x\in K^\times}J_\rho\left(\begin{bmatrix}
		x & 0 \\
		0 & 1
	\end{bmatrix}g\begin{bmatrix}
		x^{-1} & 0 \\
		0 & 1
	\end{bmatrix}\right). \label{eq:chi-cuspidal-formula}
\end{equation}
In order to apply this formula, we compute $J_\rho$ for some values of $G$. Let $f_1$ be a Bessel vector for $\rho$. Then
\[\rho\left(\begin{bmatrix}
	x & \alpha \\
	0 & x
\end{bmatrix}\right)f_1=\nu(x)\psi\left(x^{-1}\alpha\right)f_1\]
by \eqref{eq:def-cusp-b}. Hence,
\[J_\rho\left(g\begin{bmatrix}
	x & \alpha \\
	0 & x
\end{bmatrix}\right)=W_{f_1}(1)^{-1}W_{f_1}\left(g\begin{bmatrix}
	x & \alpha \\
	0 & x
\end{bmatrix}\right)=W_{f_1}(1)^{-1}W_{\rho\left(\begin{bmatrix}
	x & \alpha \\
	0 & x
\end{bmatrix}\right)f_1}(g)=\nu(x)\psi\left(x^{-1}\alpha\right)J_\rho(g).\]

The computation of $\chi_\rho$ for $c_1(x)$ is now
\[\chi_\rho\left(\begin{bmatrix}
	x & 0 \\
	0 & x
\end{bmatrix}\right)=\sum_{\alpha\in K^\times}J_\rho\left(\begin{bmatrix}
	\alpha & 0 \\
	0 & 1
\end{bmatrix}\begin{bmatrix}
	x & 0 \\
	0 & x
\end{bmatrix}\begin{bmatrix}
	\alpha^{-1} & 0 \\
	0 & 1
\end{bmatrix}\right)=(q-1)J_\rho\left(\begin{bmatrix}
	x & 0 \\
	0 & x
\end{bmatrix}\right)=(q-1)\nu(x).\]
For $c_2(x)$, we have
\begin{align*}
	\chi_\rho\left(\begin{bmatrix}
		x & 1 \\
		0 & x
	\end{bmatrix}\right) &= \sum_{\alpha\in K^\times}J_\rho\left(\begin{bmatrix}
		\alpha & 0 \\
		0 & 1
	\end{bmatrix}\begin{bmatrix}
		x & 1 \\
		0 & x
	\end{bmatrix}\begin{bmatrix}
		\alpha^{-1} & 0 \\
		0 & 1
	\end{bmatrix}\right) \\
	&= \sum_{\alpha\in K^\times}J_\rho\left(\begin{bmatrix}
		x & \alpha \\
		0 & x
	\end{bmatrix}\right) \\
	&= \nu(x)\sum_{\alpha\in K^\times}\psi\left(x^{-1}\alpha\right) \\
	&= -\nu(x).
\end{align*}
For $c_3(x,y)$, we have
\begin{align*}
	\chi_\rho\left(\begin{bmatrix}
		x & 0 \\
		0 & y
	\end{bmatrix}\right) &= \sum_{\alpha\in K^\times}J_\rho\left(\begin{bmatrix}
		\alpha & 0 \\
		0 & 1
	\end{bmatrix}\begin{bmatrix}
		x & 0 \\
		0 & y
	\end{bmatrix}\begin{bmatrix}
		\alpha^{-1} & 0 \\
		0 & 1
	\end{bmatrix}\right) \\
	&= (q-1)J_\rho\left(\begin{bmatrix}
		x & 0 \\
		0 & y
	\end{bmatrix}\right) \\
	&= (q-1)J_\rho\left(\begin{bmatrix}
		xy^{-1} & 0 \\
		0 & 1
	\end{bmatrix}\begin{bmatrix}
		y & 0 \\
		0 & y
	\end{bmatrix}\right) \\
	&= (q-1)\nu(y)J_\rho\left(\begin{bmatrix}
		xy^{-1} & 0 \\
		0 & 1
	\end{bmatrix}\right) \\
	&= 0
\end{align*}
because $x\ne y$.

The computation of $\chi_\rho(c_4(z))$ does not rely on \eqref{eq:chi-cuspidal-formula}, but rather on formulas \eqref{eq:general-cuspidal-def} and \eqref{eq:general-cuspidal-matrix}. Formula \eqref{eq:general-cuspidal-def} implies that $(k(y,x;g))_{x,y\in K^\times}$ is the matrix representing $\rho(g)$ with respect to the natural basis. Its trace is
\[\chi_\rho(g)=\sum_{x\in K^\times}k(x,x;g).\]
In particular, let $a\coloneqq z+\overline z$ and $b\coloneqq-z\overline z$. Then
\begin{align*}
	\chi_\rho(c_4(z)) &= \sum_{x\in K^\times}k\left(x,x;\begin{bmatrix}
		0 & b \\
		1 & a
	\end{bmatrix}\right) \\
	&= -q^{-1}\sum_{x\in K^\times}\psi(ax)\sum_{\substack{u\overline u=-b\\u\in L^\times}}\psi\left(-x(u+\overline u)\right)\nu(u) \\
	&= -q^{-1}\sum_{\substack{u\overline u=-b\\u\in L^\times}}\nu(u)\sum_{x\in K^\times}\psi(x(a-(u+\overline u))).
\end{align*}
If $u=z$ or $u=\overline z$, then $a=u+\overline u$; hence, $\sum_{x\in K^\times}\psi(x(a-(u+\overline u)))=q-1$. If $u\ne z$ and $u\ne\overline z$ and $u\overline u=-b$, then $a\ne u+\overline u$; hence, $\sum_{x\in K^\times}\psi(x(a-(u+\overline u)))=-1$. It follows that
\begin{align*}
	\chi_\rho(c_4(z)) &= -q^{-1}\Bigg((q-1)\nu(z)+(q-1)\nu(\overline z)-\sum_{\substack{u\ne z,\overline z\\u\overline u=-b}}\nu(u)\Bigg) \\
	&= -q^{-1}\Bigg((q\nu(z)+q\nu(\overline z))-\sum_{\substack{u\in L\\u\overline u=-b}}\nu(u)\Bigg) \\
	&= -\nu(z)-\nu(\overline z),
\end{align*}
because by \Cref{lem:sum-over-equal-norm} $\nu$ is non-decomposable and hence $\sum_{u\overline u=-b}\nu(u)=0$.

We sum up the character values computed in this section in the following table. Here, $\nu$ is a non-decomposable character of $K^\times$, and $1_{1=x}$ is the Kronecker symbol, equal to $1$ if $x=1$ and to zero if $1\ne x$. The columns gives the values of the characters of the corresponding representations.
\[\begin{array}{c|c|c|c|c}
	\text{Repr.} & c_1(x) & c_2(x) & c_3(x,y) & c_4(z) \\\hline\hline
	\rho_\nu & (q-1)\nu(x) & -\nu(x) & 0 & -\nu(z)-\nu(\overline z) \\\hline
	\rho_{(\mu_1,\mu_1)} & q\mu_1(x)^2 & 0 & \mu_1(xy) & -\mu_1(z\overline z) \\\hline
	\rho_{(\mu_1,\mu_2)} & (q+1)\mu_1(x)\mu_2(x) & \mu_1(x)\mu_2(x) & \mu_1(x)\mu_2(y)+\mu_1(y)\mu_2(x) & 0 \\\hline
	\op{Ind}^G_U\psi & (q-1)q(q+1)1_{1=x} & (1-q)1_{1=x} & 0 & 0
\end{array}\]
The columns gives the dimension and number of the representations of the corresponding type.
\[\begin{array}{c|c|c}
	\text{Repr.} & \text{Dim.} & \text{Num.} \\\hline\hline
	\rho_\nu & q-1 & \frac12\left(q^2-q\right) \\\hline
	\rho_{(\mu_1,\mu_1)} & q & q-1 \\\hline
	\rho_{(\mu_1,\mu_2)} & q+1 & \frac12(q-1)(q-2) \\\hline
	\op{Ind}^G_U\psi & (q-1)q(q+1) & 1 
\end{array}\]

\end{document}