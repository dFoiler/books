% !TEX root = ../main.tex

\documentclass[../main.tex]{subfiles}

\begin{document}

In the first three sections of this chapter, we bring all the definitions and theorems about linear representations of finite groups that we need in these notes. We refer to the appendix for proofs. The remaining two sections are devoted to a description of the group-theoretic properties of $\op{GL}(2,K)$, where $K$ is a finite field.

\section{Linear representations of finite groups} \label{sec:intro-line-reps}
Let $V$ be a finite-dimensional vector space over the field $\CC$ of the complex numbers. Denote by $\op{Aut}(V)$ the group of all automorphisms of $V$. Let $G$ be a finite group. A \textit{linear representation} of $G$ in $V$ is a homomorphism $\rho$ of $G$ into $\op{Aut}(V)$. Then $V$ is said to be the \textit{representation space} of $\rho$ and is also denoted by $V_\rho$. We shall also say that $G$ \textit{acts on $V_\rho$ through $\rho$}. The \textit{dimension} of $\rho$ is defined to be the dimension of $V_\rho$ and is denoted by $\dim\rho$. Two representations $\rho$ and $\rho'$ of $G$ are said to be \textit{isomorphic} if there exists an isomorphism $\theta\colon V_\rho\to V_{\rho'}$ such that $\theta\circ\rho(g)=\rho'(g)\circ\theta$ for every $g\in G$. We shall usually identify isomorphic representations.

A representation of $G$ of dimension $1$ is a homomorphism $\mu$ of $G$ into the multiplicative group $\CC^\times$ of $\CC$. Such a representation is called in these notes a \textit{character} of $G$. In particular, the unit character is the homomorphism of $G$ into $\CC^\times$ obtaining the value $1$ for every $g\in G$.

Let $\rho$ be a representation of $G$ and let $H$ be a subgroup of $G$. Suppose that $\mu$ is a character of $H$ for which there exists a nonzero $v\in V_\rho$ such that $\rho(h)v=\mu(h)v$ for every $h\in H$. Then $\mu$ is said to be an \textit{eigenvalue} of $H$ (with respect to $\rho$), and $v$ is said to be an \textit{eigenvector} of $H$ that belongs to $\mu$.

Again, consider a representation $\rho$ of $G$ and let $V'$ be a subspace of $V=V_\rho$ which is left invariant by $\rho(g)$ for every $g\in G$. In this case, we say that $V'$ is \textit{left-invariant} by $G$ or that $V'$ is a $G$-subspcae of $V$. Then the restriction map of $\rho(g)$ to $V'$ gives rise to a representation $\rho'$ of $G$ with $V'$ as its representation space. This representation is said to be a \textit{subrepresentation} of $\rho$, and we write $\rho'\le\rho$.

By a theorem of Maschke $V'$ has a complement in $V$, i.e., there exists another $G$-subspace $v''$ of $V$ such that $V=V'\oplus V''$. Let $\rho''$ be the corresponding subrepresentation of $\rho$. Then $\rho$ is said to be a \textit{direct sum} of $\rho'$ and $\rho''$, and we write $\rho=\rho'\oplus\rho''$. Clearly $\dim\rho=\dim\rho'+\dim\rho''$. The direct sum of $n$ representations of $G$, all isomorphic to $\rho$, is denoted by $n\rho$. A representation $\rho$ of $V$ is said to be \textit{irreducible} if its does not have a subrepresentation $\rho'$ of a lower dimension. By the theorem of Maschke, this is equivalent to saying that $\rho$ cannot be decomposed as a direct sum $\rho=\rho'\oplus\rho''$ with $\dim\rho',\dim\rho''<\dim\rho$. It follows that every representation $\rho$ of $G$ can be represented as a direct sum
\[\rho=\bigoplus_{i=1}^kn_i\rho_i,\]
where the $\rho_i$ are distinct (i.e., not-isomorphic) irreducible representations of $G$. This decomposition of $\rho$ is unique, up to the order of the summands.

There are only finitely many irreducible representations $\rho_1,\ldots,\rho_n$ of $G$. Their number $h$ is called to the number of the conjugacy classes of $G$. Their dimensions satisfy the formula
\begin{equation}
	\sum_{i=1}^n(\dim\rho_i)^2=|G|. \label{eq:dim-irred-form}
\end{equation}
If $G$ is abelian, then \eqref{eq:dim-irred-form} implies that the irreducible representations of $G$ are of dimension $1$ (i.e., they are characters) and that their number is equal to $|G|$, which in this case is the number of the conjugacy classes of $G$. Further, the set of characters of $G$ forms a multiplicative group $\widehat G$ which is isomorphic to $G$. If $1\ne\chi\in\widehat G$, then we have the following orthogonality relation:
\[\sum_{g\in G}\chi(g)=0.\]
A lemma of Artin says that the characters of $G$ are linearly independent; i.e., if $a_\chi$ are complex numbers such that $\sum_{\chi\in\widehat G}a_\chi\chi(g)=0$ for every $g\in G$, then $a_\chi=0$ for all $\chi\in G$. Now, $G$ is canonically isomorphic to the dual $\widehat{\widehat G\,}$ of $\widehat G$. Hence, the dual to this lemma is also true: if $b_g$ are complex numbers such that $\sum_{g\in G}b_g\chi(g)=0$ for every $\chi\in G$, then $b_g=0$ for all $g\in G$.

If $G$ is again an arbitrary group, then we deduce it has $[G:G^c]$ characters, where $G^c$ is the commutator subgroup of $G$. Another consequence of \eqref{eq:dim-irred-form} is that if distinct irreducible representations $\rho_1,\ldots,\rho_n$ of $G$ satisfy $\sum_{i=1}^n(\dim\rho_i)^2=|G|$, then they are all the irreducible representations of $G$.

Let $\rho$ be a representation of a finite group $G$. Then $V_\rho$ can also be considered as a module over the group ring $\CC[G]$. If $\rho'$ is an additional representation, then we write $(\rho,\rho')=(\rho,\rho')_G\coloneqq\dim\op{Hom}_{\CC[G]}(V_\rho,V_{\rho'})$. The form $(\rho,\rho')$ is clearly symmetric and bilinear with respect to direct sums. If $\rho$ and $\rho'$ are irreducible, then, by a lemma of Schur, $(\rho,\rho')=1$ if $\rho=\rho'$ and $(\rho,\rho')=0$ if $\rho\ne\rho'$. It follows that two arbitrary representations $\rho$ and $\rho'$ are \textit{disjoint}, i.e., have no common irreducible subrepresentations, if and only if $(\rho,\rho')=0$. In particular, an irreducible representation $\rho$ \textit{appears} in a representation $\rho'$, i.e., $\rho\le\rho'$, if and only if $(\rho,\rho')\ne0$; indeed, $(\rho,\rho')$ is equal to the multiplicity in which $\rho$ appears in $\rho'$.

Let $\op{End}_{\CC[G]}V_\rho\coloneqq\op{Hom}_{\CC[G]}(V_\rho,V_\rho)$. It is an algebra over $\CC$ called the \textit{Schur algebra}. If $\rho$ is irreducible, then $\op{End}_{\CC[G]}V)_{n\rho}$ is isomorphic to $M_n(\CC)$, the algebra of all $n\times n$ matrices over $\CC$. If $\rho=\bigoplus_{i=1}^kn_i\rho_i$ is the canonical decomposition of a representation $\rho$, then, by Schur's lemma,
\[\op{End}_{\CC[G]}V_\rho=\bigoplus_{i=1}^kM_{n_i}(\CC).\]
Hence, $(\rho,\rho)=\dim\op{End}_{\CC[G]}V_\rho=\sum_{i=1}^kn_i^2$. It follows that $\rho$ has no multiple components, i.e., that $n_i=1$ for all $i$, if and only if $\op{End}_{\CC[G]}V_\rho$ is commutative. In this case, $\dim\op{End}_{\CC[G]}V_\rho$ is the number of components of $\rho$.

Finally, consider a vector space $V$ of dimension $n$ over $\CC$. Every base $v_1,\ldots,v_n$ of $V$ canonically defines an isomorphism $\op{Aut}V\cong\op{GL}(n,\CC)$ (which is the group of all $n\times n$ invertible matrices over $\CC$). If $\rho\colon G\to\op{Aut}V$ is a representation of $G$, then we define $\chi_\rho(g)$ to be the trace of $\rho(g)$, where $\rho(g)$ is now considered as an element of $\op{GL}(n,\CC)$ via the above isomorphism. Clearly, $\op{tr}\rho(g)$ does not depend on the choice of the basis $v_1,\ldots,v_n$ of $V$. Hence, $\chi_\rho\colon G\to\CC$ is a well-defined function, called the \textit{character of $\rho$}. It is constant on conjugacy classes. Also, $\chi_{\rho_1\oplus\rho_2}=\chi_{\rho_1}+\chi_{\rho_2}$. Therefore, $\chi_\rho$ is said to be \textit{irreducible} if $\rho$ is irreducible. If $\dim\rho=1$, then $\chi_\rho=\chi$. In general, one defines $\dim\chi_\rho=\dim\rho$ and refers to $\chi_\rho$ as a \textit{higher-dimensional character}.

\section{Induced representations}
Let $G$ be a finite group and let $H$ be a subgroup operating on a finite-dimensional $\CC$-vector space $W$ through a representation $\tau\colon H\to\op{Aut}W$. Define a vector space $V$ to be the set of all functions $f\colon G\to W$ that satisfy
\[f(hg)=\tau(h)f(g)\qquad\text{for all }h\in H\text{ and }g\in G.\]
Thus, in order to define an element $f\in V$, it suffices to give its values on a system of representatives $G/H$ of the left classes of $G$ modulo $H$. Define an operation of $G$ on $V$ by
\[(sf)(g)\coloneqq f(gs)\qquad\text{for }s,g\in G\text{ and }f\in V.\]
The $\CC[G]$-module $V$ thus obtained is called the \textit{induced module of $W$ from $H$ to $G$} and is denoted by $\op{Ind}^G_H\tau$.

We embed $W$ in $V$ by mapping each $w\in W$ to the function $f_w\colon W\to\CC$ defined by $f_w(g)\coloneqq\tau(g)w$ if $g\in H$ and $f_w(g)=0$ if $g\in G\setminus H$. Clearly, this is a $\CC[H]$-module embedding. The image of $W$ in $V$ consists of all the functions $f\in V$ that vanish on $G\setminus H$.

Let now $G=\bigsqcup_{r\in R}rH$ be a decomposition of $G$ into left classes modulo $H$. For every $f\in V$ and for every $r\in R$, we define a function $f_r\in V$ by $f_r(g)\coloneqq f(g)$ if $g\in Hr^{-1}$ and $f_r(g)=0$ otherwise. Then $r^{-1}f_r$ belongs go $W$ (after identifying $W$ with its image in $V$), and
\[f=\sum_{r\in R}r\left(r^{-1}f_r\right).\]
Thus, $V$ is isomorphic to $\bigoplus_{r\in R}rW$. In particular, we have that $\dim V=[G:H]\dim W$.

Using this isomorphism, one obtains also a canonical isomorphism $V\cong\CC[G]\otimes_{\CC[H]}W$, where $G$ operates on the right-hand side by multiplication on the left of the first factor. This form of the induced representation is convenient to prove the following fundamental properties.
\begin{listalph}
	\item Transitivity: If $J$ is a subgroup of $H$ and $\tau\colon J\to\op{Aut}U$ is a representation of $J$, then
	\[\Ind^G_JU=\Ind^G_H\left(\Ind^H_JU\right).\]
	\item Frobenius reciprocity theorem: With the above notation, let $E$ be a $\CC[G]$-module, and denote by $\Res^G_HE$ the $\CC[H]$-module obtained from $E$ by considering only the action of $H$. Then we have the following canonical isomorphism:
	\[\op{Hom}_{\CC[G]}\left(\Ind^G_HW,E\right)\cong\op{Hom}_{\CC[H]}\left(W,\op{Res}^G_HE\right).\]
	In particular,
	\[\dim\op{Hom}_{\CC[G]}\left(\Ind^G_HW,E\right)=\dim\op{Hom}_{\CC[H]}\left(W,\op{Res}^G_HE\right).\]
	If $\tau$ and $\sigma$ are representations of $H$ and $G$ that correspond to $W$ and $E$, respectively, then the last equality can be rewritten, in the notation of \Cref{sec:intro-line-reps}, as
	\[\left(\Ind^G_H\tau,\sigma\right)_G=\left(\tau,\Res^G_H\sigma\right)_H.\]
	In particular, if both $\tau$ and $\sigma$ are irreducible, then the multiplicity of $\sigma$ in $\Ind^G_H\tau$ is equal to the multiplicity of $\tau$ in $\Res^G_H\sigma$.
\end{listalph}
Finally, if $\tau$ is a representation of a subgroup $H$ of a group $G$, and $\sigma=\Ind^G_H\tau$, then $\chi_\rho$ can be calculated from $\chi_\tau$ by the following formula
\[\chi_\rho(g)=\frac1{|H|}\sum_{r\in G}\widetilde\chi_\tau\left(sgs^{-1}\right)=\sum){r\in R}\widetilde\chi_\tau\left(rgr^{-1}\right),\]
where $\widetilde\chi_\tau$ is the function on $G$ that vanishes outside $H$ and coincides with $\chi_\tau$ on $H$; $R$ is a system of representatives of right classes of $G$ modulo $H$.

\section{The Schur algebra}
\begin{proposition} \label{prop:homs-of-ind}
	Let $H$ and $J$ be subgroups of a finite group $G$. Let $\rho$ and $\sigma$ be representations of $H$ and $J$, respectively. Then $\op{Hom}_{\CC[G]}\left(\op{Ind}^G_HV_\rho,\op{Ind}^G_JV_\sigma\right)$ is isomorphic to the vector space of all functions $F\colon G\to\op{Hom}_\CC(V_\rho,V_\sigma)$ satisfying
	\begin{equation}
		F(jgh)=\sigma(j)\circ F(g)\circ\rho(h) \label{eq:desired-schur-vec-space}
	\end{equation}
	for all $j\in J$, $g\in G$, and $h\in H$.
\end{proposition}
\begin{proof}
	Let $\widehat\rho\coloneqq\op{Ind}^G_H\rho$, $\widehat\sigma\coloneqq\op{Ind}^G_J\sigma$, and $n\coloneqq[G:H]$. Denote by $F'$ the vector space of all functions
	\[\varphi\colon G\times G\to\op{Hom}_\CC(V_\rho,V_\sigma)\]
	that satisfy
	\begin{equation}
		\varphi(jg_q,hg_2) = \sigma(j)\circ\varphi(g_1,g_2)\circ\rho(h)^{-1} \label{eq:almost-desired-schur-vec-space}
	\end{equation}
	for all $j\in J$, $h\in H$, and $g_1,g_2\in G$. For every $\varphi\in F'$, we define an element $T_\varphi\in\op{Hom}_\CC(V_{\widehat\rho},V_{\widehat\sigma})$ as follows: If $f\in V_{\widehat\rho}$, then $T_\varphi f\colon G\to V_\sigma$ is the map defined
	\begin{equation}
		(T_\varphi f)(g)\coloneqq\frac1n\sum_{r\in G}\varphi(g,r)(f(r)); \label{eq:define-desired-schur-iso}
	\end{equation}
	clearly, the map $\varphi\mapsto T_\varphi$ is a homomorphism $F'\to\op{Hom}_\CC(V_{\widehat\rho},V_{\widehat\sigma})$. It is injective. Indeed, suppose that $T_\varphi=0$. Let $s\in G$, let $v\in V_\rho$, and define a function $f_{sb}\in V_{\widehat\rho}$ by
	\[f_{sv}(g)\coloneqq\begin{cases}
		\rho(h)v & \text{if }g=hs, \\
		0 & \text{if }g\notin Hs.
	\end{cases}\]
	Then substituting $f=f_{sv}$ in \eqref{eq:define-desired-schur-iso} we have by \eqref{eq:almost-desired-schur-vec-space} that $\varphi(g,s)v=0$. Hence, $\varphi(g,s)=0$; i.e., $\varphi=0$.

	The dimension of $F'$ is equal to $[G:H][G:J](\dim\rho)(\dim\sigma)$ by \eqref{eq:almost-desired-schur-vec-space}. This is also the dimension of $\op{Hom}_\CC(V_{\widehat\rho},V_{\widehat\sigma})$. Hence, $T$ is an isomorphism.

	Denote now by $F'_G$ the subspace of all $\varphi\in F'$ such that $T_\varphi\in\op{Hom}_{\CC[G]}(V_{\widehat\rho},V_{\widehat\sigma})$. Clearly $\varphi\in F'_G$ if and only if
	\begin{equation}
		\sum_{r\in G}\varphi\left(g,rx^{-1}\right)(f(r))=\sum_{r\in G}\varphi(gx,r)(f(r)) \label{eq:inv-in-almost-desired-schur-vec-space}
	\end{equation}
	for all $f\in V_{\widehat\rho}$ and $x\in G$. Substituting $f=f_{sv}$ in \eqref{eq:inv-in-almost-desired-schur-vec-space}, we have that \eqref{eq:inv-in-almost-desired-schur-vec-space} is equivalent to the condition
	\begin{equation}
		\varphi\left(g,rx^{-1}\right)=\varphi(gx,r)\qquad\text{for all }g,r,x\in G. \label{eq:better-inv-in-almost-desired-schur-vec-space}
	\end{equation}

	For every function $F\colon\op{Hom}_\CC(V_\rho,V_\sigma)$ that satisfies \eqref{eq:desired-schur-vec-space}, we define a function $\varphi\colon G\times G\to\op{Hom}_\CC(V_\rho,V_\sigma)$ by
	\begin{equation}
		\varphi(g_1,g_2)\coloneqq F\left(g_1g_2^{-1}\right). \label{eq:to-desired-schur-vec-space}
	\end{equation}
	Then $\varphi$ satisfies \eqref{eq:better-inv-in-almost-desired-schur-vec-space}, and thus it belongs to $F'_G$. Conversely, starting from $\varphi$ in $F'_G$, we define an $F\colon G\to\op{Hom}_\CC(V_\rho,V_\sigma)$ by
	\[F(g)\coloneqq\varphi(g,1).\]
	Then $F$ satisfies \eqref{eq:desired-schur-vec-space}, and the $\varphi$ defined by \eqref{eq:to-desired-schur-vec-space} coincides with the one we started with. Thus, $F$ is isomorphic to $F'_G$.

	For every $F\in F$, denote by $T_F$ the element of $\op{Hom}_{\CC[G]}(V_{\widehat\rho},V_{\widehat\sigma})$ defined by
	\begin{equation}
		(T_Ff)(g)\coloneqq\frac1n\sum_{r\in G}F\left(gr^{-1}\right)(f(r)). \label{eq:desired-schur-iso}
	\end{equation}
	Then the map $F\mapsto T_F$ is the desired isomorphism.
\end{proof}
\begin{corollary}
	In the notation of \Cref{prop:homs-of-ind}, we have
	\[\left(\op{Ind}^G_H,\op{Ind}^G_J\sigma\right)\le|J\backslash G/H|(\dim\rho)(\dim\sigma)\]
	where $J\backslash G/H$ denotes the set of double classes of $G$ modulo $J$ and $H$.
\end{corollary}

The most interesting conclusion of \Cref{prop:homs-of-ind} arises in the special case where $H=J$ and $\rho=\sigma$. In this case $\op{Hom}_{\CC[G]}\left(\op{Ind}^G_HV_\rho,\op{Ind}^G_JV_\sigma\right)=\op{End}_{\CC[G]}(V_{\widehat\rho})$, the Schur algebra of $\widehat\rho$. The bijection between $F$ and this algebra established in \Cref{prop:homs-of-ind} turns $F$ into an algebra and the product between two elements $F_1$ and $F_2$ of $F$ is given by
\begin{equation}
	(F_1*F_2)(g)\coloneqq\frac1{[G:H]}\sum_{s\in G}F_1\left(gs^{-1}\right)F_2(s). \label{eq:better-schur-ind-mult}
\end{equation}
This can be easily verified from the basic relation $T_{F_1}T_{F_2}=T_{F_1*F_2}$ and the definition \eqref{eq:desired-schur-iso}.

\section{The group \texorpdfstring{$\GL(2,K)$}{ GL(2,K)}}
In this section, we fix our notation for the rest of these notes.

Let $K$ be a finite field with $q$ elements and suppose $q>2$. We denote by $G$ the group $\op{GL}(2,K)$ of all $2\times2$ invertible matrices with entries in $K$. We further reserve some letters for distinguished subgroups of $G$ that will concern us in the sequel. The letter $B$ stands for the \textit{Borel} subgroup of $G$ consisting of all upper triangular matrices
\[B\coloneqq\left\{\begin{bmatrix}
	\alpha & \beta \\
	0 & \delta
\end{bmatrix}:\alpha,\delta\in K^\times,\beta\in K\right\}.\]
Clearly $|B|=(q-1)^2q$. Straightforward calculations show that the matrix $w\coloneqq\begin{bmatrix}
	0 & 1 \\
	1 & 0
\end{bmatrix}$, together with the matrices $\begin{bmatrix}
	1 & 0 \\
	\gamma & 1
\end{bmatrix}$, $\gamma\in K$, form a system of representatives for the left (and also for the right) classes of $G$ modulo $B$. Hence, $[G:B]=q+1$ and thus $|G|=(q-1)^2q(q+1)$. The idempotent matrix $w$ will play an important role in the sequel.

Note $B$ is a solvable group. Indeed, $B$ contains the normal abelian subgroup
\[U\coloneqq\left\{\begin{bmatrix}
	1 & \beta \\
	0 & 1
\end{bmatrix}:\beta\in K\right\}\]
of all unipotent upper-triangular matrices. This group is isomorphic to the additive group $K^+$ of the field $K$. Indeed,
\[\begin{bmatrix}
	1 & \beta \\
	0 & 1
\end{bmatrix}\begin{bmatrix}
	1 & \beta' \\
	0 & 1
\end{bmatrix}=\begin{bmatrix}
	1 & \beta+\beta' \\
	0 & 1
\end{bmatrix}.\]
We shall therefore sometimes identify an element $\beta$ of $K$ with the corresponding matrix of $U$. The quotient group $B/U$ is isomorphic to the \textit{Cartan} group
\[D\coloneqq\left\{\begin{bmatrix}
	\alpha & 0 \\
	0 & \delta
\end{bmatrix}:\alpha,\delta\in K^\times\right\}\]
of all diagonal matrices. It is isomorphic to $K^\times\times K^\times$ and hence is abelian. Clearly, $U\cap D=1$ and $UD=B$. Hence, $B$ is the semi-direct product of $U$ by $D$. Simple calculation shows that $U$ is the commutator subgroup of $B$. (Here we are using the assumption $q>2$. In the case $q=2$, we have $B=U$ and $B^c=1$.) In particular, it follows that $B$ has exactly $(q-1)^2$ characters.

Another important normal subgroup of $B$ is
\[P\coloneqq\left\{\begin{bmatrix}
	\alpha & \beta \\
	0 & 1
\end{bmatrix}:\alpha\in K^\times,\beta\in K\right\}\]
of order $(q-1)q$ and of index $q-1$ in $B$. The center
\[Z\coloneqq\left\{\begin{bmatrix}
	\delta & 0 \\
	0 & \delta
\end{bmatrix}:\delta\in K^\times\right\}\]
of $G$ is also contained in $B$. Clearly $Z\cap P=1$ and $ZP=B$; i.e., $B$ is the semi-direct product of $Z$ and $P$.

Note that $U$ is contained in $P$. In fact, $U$ is also the commutator subgroup of $P$. A complement of $U$ in $P$ is the group
\[A\coloneqq\left\{\begin{bmatrix}
	\alpha & 0 \\
	0 & 1
\end{bmatrix}:\alpha\in K^\times\right\},\]
which is canonically isomorphic to $K^\times$. Thus, $P$ is the semi-direct product of $U$ by $A$. The action of $A$ on $U$ by conjugation corresponds to the action of $K^\times$ on $K^+$ by multiplication
\[\begin{bmatrix}
	\alpha & 0 \\
	0 & 1
\end{bmatrix}\begin{bmatrix}
	1 & \beta \\
	0 & 1
\end{bmatrix}\begin{bmatrix}
	\alpha^{-1} & 0 \\
	0 & 1
\end{bmatrix}=\begin{bmatrix}
	1 & \alpha\beta \\
	0 & 1
\end{bmatrix}.\]

Our method of constructing the representations of $G$ consists of three stages: First of all we use general principles and easily determine the representations of $P$. Then we make a jump to $B$ and induce characters from $B$ to $G$. The last and most difficult stage is to explore those representations of $G$ that do not appear in the former stage. In doing this we shall use the \textit{Bruhat decomposition} of $G$, namely $G=B\sqcup BwU$. Indeed, if $\gamma\ne0$, then
\[\begin{bmatrix}
	\alpha & \beta \\
	\gamma & \delta
\end{bmatrix}=\begin{bmatrix}
	\beta-\alpha\gamma^{-1}\delta & \alpha \\
	0 & \gamma
\end{bmatrix}\begin{bmatrix}
	0 & 1 \\
	1 & 0
\end{bmatrix}\begin{bmatrix}
	1 & \gamma^{-1}\delta \\
	0 & 1
\end{bmatrix}.\]

\section{The conjugacy classes of \texorpdfstring{$\GL(2,K)$}{ GL(2,K)}}
Before we start to investigate the irreducible representations of $G$, we would like to compute their number. It is equal to the number of the conjugacy classes of $G$. The computation of this number will be done by explicitly giving a representative for each of the conjugacy classes. This will also help us later to give the \textit{character table} of $G$, i.e., the values of the irreducible higher-dimensional characters at the conjugacy classes.

An element $g$ of $G$ has two eigenvalues. If one of them belongs to $K$, then so does the other, since they both the same quadratic equation, $\deg(g-XI)=0$ over $K$. All the elements in the conjugacy class of $G$ have the same eigenvalues. There are therefore two possibilities.
\begin{listalph}
	\item The eigenvalues of $g$ belong to $K$.

	In this case, $g$ is conjugate over $K$ to a unique matrix in a canonical Jordan form. If both eigenvalues are equal to the same element $\alpha$ of $K$, then the Jordan form is
	\[c_1(\alpha)\coloneqq\begin{bmatrix}
		\alpha & 0 \\
		0 & \alpha
	\end{bmatrix}\qquad\text{or}\qquad c_2(\alpha)\coloneqq\begin{cases}
		\alpha & 1 \\
		0 & \alpha
	\end{cases},\]
	depending on whether the minimal polynomial of $g$ is different from the characteristic polynomial or equal to it. If the eigenvalues are $\alpha,\beta$ and $\alpha\ne\beta$, then the Jordan form is
	\[c_3(\alpha,\beta)\coloneqq\begin{cases}
		\alpha & 0 \\
		0 & \beta
	\end{cases}.\]
	There are $q-1$ matrices of the form $c_1(\alpha)$, $q-1$ of the form $c_2(\alpha)$, and $\frac12(q-1)(q-2)$ of the form $c_3(\alpha,\beta)$.

	\item The eigenvalues of $g$ do not belong to $K$.

	In this case, they belong the unique quadratic extension $L$ of $K$. Denote by $p(X)$ the characteristic polynomial of $g$. Then $p(X)$ is irreducible over $K$, and its roots $\alpha,\overline\alpha$, which are the eigenvalues of $g$, are conjugate over $K$. They are distinct, since $K$ as a finite field is perfect. If we denote $\op{Tr}(\alpha)\coloneqq\alpha+\overline\alpha$ and $\op N(\alpha)\coloneqq\alpha\overline\alpha$, then $p(X)=X^2-\op{Tr}(\alpha)X+\op N(\alpha)$.

	Let $v$ be a nonzero vector in $K^2$. Then $v,gv$ form a basis for $K^2$ over $K$, since otherwise there would exist a $\lambda\in K$ such that $gv=\lambda v$. This $\lambda$ would then be an eigenvalue, contrary to our hypothesis. Recalling that $p(g)=0$ (by the Cayley--Hamilton theorem), we have that the matrix of $g$, when considered as a linear operator on $K^2$ with respect to the basis $v,gv$ is
	\[c_4(\alpha)\coloneqq\begin{bmatrix}
		0 & -\op N(\alpha) \\
		1 & \op{Tr}(\alpha)
	\end{bmatrix}.\]
	Thus, $g$ is conjugate in $G$ to $c_4(\alpha)$.

	Conversely, given an $\alpha\in L\setminus K$, then $c_4(\alpha)$ is a matrix in $G$ with the eigenvalues $\alpha,\overline\alpha$. If $\beta$ is an additional element of $L\setminus K$, then $c_4(\alpha)$ is conjugate to $c_4(\beta)$ if and only if $\beta=\alpha$ or $\beta=\overline\alpha$, since then $p(\beta)=0$.

	There are $q^2-q$ elements in $L\setminus K$. Hence, there are $\frac12\left(q^2-q\right)$ matrices of the form $c_4(\alpha)$.
\end{listalph}
We sum up our results in the following.
\begin{proposition}
	The conjugacy classes of $G$ are classified in four families.
	\begin{listalph}
		\item $q-1$ classes, represented by $c_1(\alpha)$, with equal eigenvalues in $K$ such that the characteristic polynomial is different from the minimal polynomial;
		\item $q-1$ classes, represented by $c_2(\alpha)$, with equal eigenvalues in $K$ such that the characteristic polynomial is equal to the minimal polynomial;
		\item $\frac12(q-1)(q-2)$ classes, represented by $c_3(\alpha,\beta)$, with distinct eigenvalues in $K$;
		\item $\frac12\left(q^2-q\right)$ classes, represented by $c_4(\alpha)$, with eigenvalues in $L\setminus K$.
	\end{listalph}
\end{proposition}

\end{document}