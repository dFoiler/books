% !TEX root = ../main.tex

\documentclass[../main.tex]{subfiles}

\begin{document}

This chapter starts with the representations of $P$, then investigates the behavior of representations of $G$ that are induced from characters of $B$, and finally describes the cuspidal representations of $G$, i.e., those representations that do not appear as components of the induced ones. The chapter ends with Weil's reciprocity law.

\section{The representations of \texorpdfstring{$P$}{ P}} \label{sec:reps-of-p}
We use the method of ``small groups'' of Wigner in order to determine the representations of $P$.%\todo{}

First, we fix for the rest of these notes a non-unit character $\psi$ of $K^+$. We consider it also as a character of $U$. For every $a\in A$, we define a character $\psi_a$ of $U$ by
\begin{equation}
	\psi_a(u)\coloneqq\psi\left(aua^{-1}\right),\qquad\text{for }u\in U. \label{eq:def-psi-a}
\end{equation}
If $a\ne a'$, then $\psi_a\ne\psi_{a'}$. Indeed, if
\[a\coloneqq\begin{bmatrix}
	\alpha & 0 \\
	0 & 1
\end{bmatrix},\qquad a'\coloneqq\begin{bmatrix}
	\alpha' & 0 \\
	0 & 1
\end{bmatrix},\qquad\text{and}\qquad u\coloneqq\begin{bmatrix}
	1 & \beta \\
	0 & 1
\end{bmatrix},\]
then $\psi_a(u)=\psi(\alpha\beta)$, and $\psi_a=\psi_{a'}$ implies that $\psi((\alpha-\alpha')\beta)=0$ for all $\beta\in K$; hence $\alpha=\alpha'$; hence $a=a'$. We thus get $q-1$ distinct representatives of $U$. These, together with the unit representation of $U$, are all the characters of $U$ since $\left|U\right|=q$.

Every character $\chi$ of $A$ can be lifted to a character $\widetilde\chi$ of $P$ defined by $\widetilde\chi(ua)\coloneqq\chi(a)$. The $q-1$ distinct characters $\widetilde\chi$ of $P$ obtained in this way are all the characters of $P$ since $\left[P:P^c\right]=[P:U]=q-1$.

In order to find the higher-dimensional representations of $P$, we induce $\psi$ from $U$ to $P$ and claim
\begin{equation}
	\op{Res}^P_U\op{Ind}^P_U\psi\stackrel?=\bigoplus_{a\in A}\psi_a. \label{eq:res-ind-psi}
\end{equation}
Indeed, for every $a\in A$, we define a function $f_a\in\op{Ind}^P_UV_\psi$ by
\[f_a(a')\coloneqq\begin{cases}
	1 & \text{if }a=a', \\
	0 & \text{if }a\ne a',
\end{cases}\qquad\text{where }a'\in A.\]
Then $f_a$ is an eigenvector of $U$ that belongs to the eigenvalue $\psi_a$. In order to prove this claim, we have to show that $f_a(pu)=\psi_a(u)f_a(p)$ for every $p\in P$ and every $u\in U$. Writing $p=u'a'$ with $u'\in U$ and $a'\in A$ and using the identity $f_a(u'p')=\psi(u')f_a(p')$ for $p'\in P$, we see that it suffices to show that
\begin{equation}
	f_a(a'u)\stackrel?=\psi_a(u)f_a(a'). \label{eq:almost-f-a-eigenvector}
\end{equation}
Indeed,
\[f_a(a'u)=f_a\left(a'u(a')^{-1}a'\right)=\psi\left(a'u(a')^{-1}\right)f_a(a')=\psi_{a'}(u)f_a(a').\]
The right-hand side is equal to zero if $a\ne a'$ and equal to $\psi_a(u)f_a(a')$ if $a=a'$; hence, \eqref{eq:almost-f-a-eigenvector} is true in both cases. Thus, the vector $f_a$ generates the one-dimensional space $V_{\psi_a}$.

If we let $a$ vary, we get $q-1$ linearly independent vectors $f_a$ of the $(q-1)$-dimensional vector space $\op{Ind}^P_UV_\psi$. Hence, $\op{Res}^P_U\op{Ind}^P_UV_\psi=\bigoplus_{a\in A}V_{\psi_a}$ as $U$-modules. This proves \eqref{eq:res-ind-psi}.

As a consequence of \eqref{eq:res-ind-psi}, we prove the following fundamental theorem.
\begin{theorem} \label{thm:reps-of-p}
	The group $P$ has $q$ irreducible representations.
	\begin{listalph}
		\item $(q-1)$ of them are one-dimensional; they are the lifting of the characters of $A$;
		\item one $(q-1)$-dimensional representation which is $\pi\coloneqq\op{Ind}^P_U\psi$.
	\end{listalph}
\end{theorem}
\begin{proof}
	We only have to prove (b). First, note that $\dim\op{Ind}^P_U\psi=[P:U]=q-1$. Second, by the Frobenius reciprocity theorem and by \eqref{eq:res-ind-psi},
	\[\left(\op{Ind}^P_U\psi,\op{Ind}^P_U\psi\right)_P=\left(\psi,\bigoplus_{a\in A}\psi_a\right)=1;\]
	hence $\op{Ind}^P_U\psi$ is an irreducible character of $P$.

	In order to prove that there is no additional representation of $P$, one can observe that
	\[\sum_{a\in A}(\dim\psi_a)^2+\left(\dim\op{Ind}^P_U\psi\right)^2=(q-1)+(q-1)^2=\left|P\right|.\]

	We would however also like to prove the last assertion without using the finiteness of $G$. In order to do this, note first that one can in fact replace $\psi$ in \eqref{eq:res-ind-psi} by $\psi_{a'}$ and have
	\begin{equation}
		\op{Res}^P_U\op{Ind}^P_U\psi_{a'}=\bigoplus_{a\in A}\psi_{a}. \label{eq:res-ind-psi-a}
	\end{equation}
	Hence, we can prove, as before, that $\op{Ind}^P_U\psi_{a'}$ is an irreducible representation of $P$. Further, by \eqref{eq:res-ind-psi-a},
	\[\left(\op{Ind}^P_U\psi,\op{Ind}^P_U\psi_{a'}\right)=\left(\psi,\bigoplus_{a\in A}\psi_a\right)=1.\]
	Hence,
	\begin{equation}
		\op{Ind}^P_U\psi=\op{Ind}^P_U\psi_{a'}. \label{eq:ind-psi-is-ind-psi-a}
	\end{equation}
	Now, let $\sigma$ be an arbitrary irreducible representation of $P$ and consider $\op{Res}^P_U\sigma$. If there exists an $a'\in A$ such that $\psi_{a'}$ appears in $\op{Res}^P_U\sigma$, then by \eqref{eq:ind-psi-is-ind-psi-a},
	\[\left(\sigma,\op{Ind}^P_U\right)=\left(\op{Res}^P_U\sigma,\psi_{a'}\right)>0.\]
	Hence, $\sigma=\op{Ind}^P_U\psi$ because both representations are irreducible. Otherwise, $\op{Res}^P_U\sigma$ is a multiple of the unit character of $U$; i.e., $\sigma(u)v=v$ for every $v\in V_\sigma$. Consider therefore $\op{Res}^P_AV_\sigma$. It decomposes into linear $A$-subspaces because $A$ is abelian. In particular, there exists a nonzero vector $v\in V_\sigma$ and a character of $A$, say $\chi$, such that $\sigma(a)v=\chi(a)v$ for every $a\in A$. Hence, if $u\in U$, then $\sigma(au)v=\chi(a)v$. It follows that $\sigma=\widetilde\chi$.
\end{proof}
\begin{remark}
	Note that we have also proved that our description of the representations of $P$ is independent of the choice of $\psi$.
\end{remark}
\begin{remark}
	The distinguished representation $\pi=\op{Ind}^P_U\psi$ will play an important rule in the sequel. This is the reason for reserving the letter $\pi$ for it.
\end{remark}

\section{The representations of \texorpdfstring{$B$}{ B}}
We have already that the commutator of $B$ is $U$ (see \cref{sec:gl-2-k-struct}). Moreover, $B$ is the semi-direct product of $D$ and $U$. The group $D$ is canonically isomorphic to $K^\times\times K^\times$. Hence, every pair $(\mu_1,\mu_2)$ of characters of $K^\times$ defines a unique character $\mu$ of $B$ which is given by the formula
\begin{equation}
	\mu\left(\begin{bmatrix}
		\alpha & \beta \\
		0 & \delta
	\end{bmatrix}\right) \coloneqq\mu_1(\alpha)\mu_2(\beta),\qquad\alpha,\delta\in K^\times. \label{eq:def-char-on-b}
\end{equation}
Conversely, to every character $\mu$ of $B$, there corresponds a pair of characters $(\mu_1,\mu_2)$ of $K^\times$ such that \eqref{eq:def-char-on-b} holds. Thus, the $(q-1)^2$ characters of $B$ given by \eqref{eq:def-char-on-b} are all the characters of $B$.

An easy computation shows that the normalizer of $D$ in $G$ is generated by $D$ and $w$.\footnote{This needs $q>2$.} Indeed, we have
\[w\begin{bmatrix}
	\alpha & 0 \\
	0 & \delta
\end{bmatrix}w^{-1}=\begin{bmatrix}
	\delta & 0 \\
	0 & \alpha
\end{bmatrix}.\]
For every character $\mu$ of $B$ given by \eqref{eq:def-char-on-b}, we define a character $\mu_w$ of $B$ by
\[\mu_w\left(\begin{bmatrix}
	\alpha & \beta \\
	0 & \delta
\end{bmatrix}\right)\coloneqq\mu_2(\delta)\mu_2(\alpha).\]
Then $\mu_w(d)=\mu\left(wdw^{-1}\right)$ for every $d\in D$, and
\[\mu_w=\mu\iff\mu_1=\mu_2.\]
In order to find the higher-dimensional representations of $B$, recall that $B$ is also the semi-direct product of $Z$ by $P$. The abelian group $Z$ has $q-1$ characters $\chi$. Each of them can be extended to a character $\widetilde\chi$ of $B$ by
\[\widetilde\chi(zp)\coloneqq\chi(z)\qquad\text{for }z\in Z,p\in P.\]
Also, composing the canonical map $B\to P$ with kernel $Z$, with the representation $\pi$ of $B$ (see \Cref{thm:reps-of-p}), we get an irreducible representation $\widetilde\pi$ of $B$ of dimension $q-1$:
\[\widetilde\pi(zp)\coloneqq\pi(p)\qquad\text{for }z\in Z,p\in P.\]
The tensor product
\begin{equation}
	(\widetilde\chi\otimes\widetilde\pi)(zp)\coloneqq\chi(z)\pi(p),\qquad\text{for }z\in Z,p\in P \label{eq:twist-pi-to-b}
\end{equation}
is an irreducible $(q-1)$-dimensional representation of $B$ whose restriction to $Z$ is $\chi$. Varying $\chi$ on all characters of $Z$, we get $q-1$ different $(q-1)$-dimensional representations of $B$. These, together with the $(q-1)^2$ characters of $B$, are all the irreducible representations of $B$ because
\[(q-1)^2+(q-1)(q-1)^2=q(q-1)^2=\left|B\right|.\]
We have therefore proved the following theorem.
\begin{theorem}
	The group $B$ has the following irreducible representations.
	\begin{listalph}
		\item $(q-1)^2$ characters by \eqref{eq:def-char-on-b}.
		\item $(q-1)$ different $(q-1)$-dimensional representations given by \eqref{eq:twist-pi-to-b}.
	\end{listalph}
\end{theorem}

\section{Inducing characters from \texorpdfstring{$B$}{ B} to \texorpdfstring{$G$}{ G}}
As a first step toward the determination of the irreducible representations of $G$, we investigate those that appear as components of $\op{Ind}^G_B\mu$, where $\mu$ is a character of $B$. In order to shorten the notation, we make the convention
\[\widehat\mu\coloneqq\op{Ind}^G_B\mu\]
and stick to it for the rest of these notes. The dimension of $\widehat\mu$ is $q+1$. Our task in this section is to determine the connection between $\mu$ and $\widehat\mu$.

In order to do this, we have the following definition.
\begin{definition}[Jacquet module]
	We define the \textit{Jacquet module} of a representation $\rho$ of $G$ as
	\[J(V_\rho)\coloneqq\{v\in V_\rho:\rho(u)v=v\text{ for every }u\in U\}.\]
\end{definition}
The fact that $U$ is normal in $B$ implies that $B$ acts on $J(V_\rho)$. Indeed, if $v\in J(V_\rho)$, $b\in B$, and $u\in U$, then $b^{-1}ub\in U$; hence
\[\rho(u)\rho(b)v=\rho(b)\rho\left(b^{-1}ub\right)v=\rho(b)v.\]
It might happen however that $J(V_\rho)$ is not a $G$-invariant subspace.

If $\rho_1$ and $\rho_2$ are two representations of $\rho$, then clearly
\[J\left(V_{\rho_1}\oplus V_{\rho_2}\right)=J\left(V_{\rho_1}\right)\oplus J\left(V_{\rho_2}\right).\]
In particular, we have the following lemma.
\begin{lemma} \label{lem:jac-of-mu-hat}
	If $\mu$ is a character of $B$, then $\dim J(V_{\widehat\mu})=2$.
\end{lemma}
\begin{proof}
	By definition, $J(V_{\widehat\mu})$ contains all functions $f\colon G\to\CC$ that satisfy
	\[f(bg)=\mu(B)f(g)\qquad\text{and}\qquad f(bu)=f(b)\]
	for all $b\in B$, $g\in G$, and $u\in U$. In particular,
	\[f(b)=\mu(b)f(1)\qquad\text{and}\qquad f(bwu)=\mu(b)f(w).\]
	Using the Bruhat decomposition $G=B\sqcup BwU$, this implies that $f$ is determined by its values in $1$ and in $w$. It follows that $\dim J(V_{\widehat\mu})=2$, and a canonical basis for $J(V_{\widehat\mu})$ is the two functions $f_1$ and $f_2$ satisfying
	\begin{equation}
		\left\{\begin{array}{c}
			f_1(1)=1, \\
			f_1(w)=0,
		\end{array}\right.\qquad\text{and}\qquad\left\{\begin{array}{c}
			f_2(1)=0, \\
			f_2(w)=1.
		\end{array}\right. \label{eq:jac-of-mu-hat-basis}
	\end{equation}
	This completes the proof.
\end{proof}
A supplement to \Cref{lem:jac-of-mu-hat} is the following.
\begin{lemma} \label{lem:eigens-of-jac-mu-hat}
	If $\mu$ is a character of $B$, then $B$ operating on $J(\mu_{\widehat\mu})$ has two eigenvectors $f_1$ and $f_2$ (defined by \eqref{eq:jac-of-mu-hat-basis}) that correspond to the eigenvalues $\mu$ and $\mu_w$, respectively. In detail,
	\begin{equation}
		\widehat\mu(b)f_1=\mu(b)f_1\qquad\text{and}\qquad\widehat\mu(b)f_2=\mu_w(b)f_2 \label{eq:desired-eigen-of-jac-mu-hat}
	\end{equation}
	for every $b\in B$.
\end{lemma}
\begin{proof}
	We have only to prove that both sides of the equalities \eqref{eq:desired-eigen-of-jac-mu-hat} coincide in $1$ and in $w$.

	Indeed, for $f_1$, we have $\left(\widehat\mu(b)f_1\right)(1)=f_1(b)=\mu(b)f_1(1)$. Also, by the Bruhat decomposition, there exists for every $b\in B$ elements $b_1\in B$ and $u\in U$ such that $wb=b_1wu$. Hence,
	\[\left(\widehat\mu(b)f_1\right)(w)=f_1(wb)=f_1(b_1wu)=\mu(b_1)f_1(w)=0=\mu(b)f_1(w).\]
	For $f_2$, we have
	\[\left(\widehat\mu(b)f_2\right)(1)=f_2(b)=\mu(b)f_2(1)=0=\mu_w(b)f_2(1).\]
	A difficulty arises in calculating $\left(\widehat\mu(b)f_2\right)(w)$. We overcome this by first considering $d\in D$. Then
	\begin{equation}
		\left(\widehat\mu(d)f_2\right)(w)=f_2(wd)=f_2(wdww)=\mu(wdw)f_2(w)=\mu_w(d)f_2(w). \label{eq:desired-eigen-of-jac-mu-hat-d}
	\end{equation}
	In general, we know from \Cref{lem:jac-of-mu-hat} that $f_1$ and $f_2$ generate $J(V_{\widehat\mu})$. Hence, for every $b\in B$, there exist $\alpha_1(b),\alpha_2(b)\in\CC$ such that
	\begin{equation}
		\widehat\mu(b)f_2=\alpha_1(b)f_1+\alpha_2(b)f_2. \label{eq:decompose-b-f2}
	\end{equation}
	Calculating both sides of \eqref{eq:decompose-b-f2} at $1$, we obtain $\alpha_1(b)=0$; hence, $\widehat\mu(b)f_2=\alpha_2(b)f_2$. It follows that $\alpha_2(b_1b_2)=\alpha_2(b_1)\alpha_2(b_2)$ for $b_2,b_2\in B$; i.e., $\alpha-2$ is a character of $B$. Hence, if
	\[b\coloneqq\begin{bmatrix}
		\alpha & \beta \\
		0 & \delta
	\end{bmatrix}\qquad\text{and}\qquad d\coloneqq\begin{bmatrix}
		\alpha & 0 \\
		0 & \delta
	\end{bmatrix},\]
	then
	\[\left(\widehat\mu(b)f_2\right)(w)=\alpha_2(b)f_2(w)=\alpha_2(d)f_2(w)=\left(\widehat\mu(d)f_2\right)(w)=\mu_w(d)f_2(w)=\mu_w(b)f_2(w),\]
	which completes the proof.
\end{proof}
The importance of the Jacquet modules for our investigation lies in the following lemma.
\begin{lemma} \label{lem:use-of-jac}
	Let $\rho$ be a representation of $G$. Then $J(V_\rho)\ne0$ if and only if there exists character $\mu$ of $B$ such that $\left(\rho,\widehat\mu\right)\ne0$.
\end{lemma}
\begin{proof}
	Suppose $J(V_\rho)\ne0$. Then $J(V_\rho)$ can be considered as a nontrivial $B/U$-space via $\rho$. But $B/U$ is abelian; hence $J(V_\rho)$ splits into a direct sum of one-dimensional $B/U$-subspaces. It follows that there exists a character $\mu$ of $B$ and a nonzero element $v\in J(V_\rho)$ such that $\rho(b)v=\mu(b)v$ for every $b\in B$. Hence, $\left(\op{Res}^G_B\rho,\mu\right)\ne0$. By the Frobenius reciprocity theorem, $(\rho,\widehat\mu)\ne0$, and half of the lemma is thus proved.

	Now, suppose that $(\rho,\widehat\mu)\ne0$. Then arguing backwards we find that there exists a nonzero element $v\in V_\rho$ such that $\rho(b)v=\mu(b)v$ for every $b\in B$. Now, $\mu$ is trivial on $U$ because $U$ is the commutator of $B$. Hence, $v\in J(V_\rho)$, which is therefore not zero.
\end{proof}
\begin{corollary} \label{cor:irred-use-of-jac}
	Let $\rho$ be an irreducible representation of $G$. Then $J(V_\rho)\ne0$ if and only if there exists a character $\mu$ of $B$ such that $\rho\le\widehat\mu$.
\end{corollary}
\begin{proof}
	Immediate from \Cref{lem:use-of-jac}.
\end{proof}
\begin{corollary} \label{cor:mu-hat-decomp}
	If $\mu$ is a character of $B$, then $\widehat\mu$ has at most two irreducible components.
\end{corollary}
\begin{proof}
	Let $\widehat\mu=\rho_1\oplus\cdots\oplus\rho_r$ be a decomposition of $\widehat\mu$ into irreducible components. Then
	\[J(V_{\widehat\mu})=J\left(V_{\rho_1}\right)\oplus\cdots\oplus J\left(V_{\rho_r}\right).\]
	By \Cref{cor:irred-use-of-jac}, $J\left(V_{\rho_i}\right)\ne0$ for $i=1,\ldots,r$. Hence, the dimension of the right-hand side is $\ge r$. On the other hand, $\dim J\left(V_{\widehat\mu}\right)=2$ by \Cref{lem:jac-of-mu-hat}. Hence, $r\le2$.
\end{proof}
The next few lemmas give the exact information about the possible two components of $\widehat\mu$.
\begin{lemma} \label{lem:to-reducible}
	If $\mu$ is a character of $B$ and $\mu=\mu_w$, then $\widehat\mu$ has a one-dimensional component.
\end{lemma}
\begin{proof}
	The assumption implies that $\mu$ corresponds to a pair $(\mu_1,\mu_1)$ of characters $K^\times$; i.e., if
	\[b\coloneqq\begin{bmatrix}
		\alpha & \beta \\
		0 & \delta
	\end{bmatrix},\]
	then $\mu(b)=\mu_1(\alpha)\mu_1(\delta)$. It follows that $\mu(b)=\mu_2(\det b)$. Now, define a function $f\colon G\to\CC$ by $f(g)\coloneqq\mu_1(\deg g)$. Then $f(bg)=\mu(b)f(g)$, that is, $f$ belongs to $V_{\widehat\mu}$. Moreover,
	\[\left(\widehat\mu(s)f\right)(g)=f(gs)=\mu_1(\det s)f(g)\]
	for $s,g\in G$. It follows that $f$ is an eigenvector of $G$ that belongs to the eigenvalue $\mu_1\circ\det$.
\end{proof}
\begin{lemma}
	If $\mu$ is a character of $B$, then $\widehat\mu$ has at most one one-dimensional component.
\end{lemma}
\begin{proof}
	Assume that $\widehat\mu$ has two distinct one-dimensional components, $\chi_1$ and $\chi_2$. Then by \Cref{cor:mu-hat-decomp} they are all the components of $\widehat\mu$; i.e., $\widehat\mu=\chi_1\oplus\chi_2$. It follows that $q+1=\dim\widehat\mu=2$, which is a contradiction.
\end{proof}
\begin{lemma} \label{lem:decomp-res-mu-hat}
	If $\mu$ is a character of $B$, then
	\[\op{Res}_PV_{\widehat\mu}=\op{Res}_PJ(V_{\widehat\mu})\oplus V_\pi.\]
\end{lemma}
\begin{proof}
	Note $J(V_{\widehat\mu})$ is a two-dimensional $B$-subspace of $V_{\widehat\mu}$; in particular, $J(V_{\widehat\mu})$ is a $P$-subspace of $V_{\widehat\mu}$. Let $V$ be a $P$-complement to $J(V_{\widehat\mu})$ in $V_{\widehat\mu}$. Then $\dim V=q-1$. Further, $V$ has no one-dimensional $P$-subspace; indeed otherwise, there would exist a nonzero element $v\in V$ and a character $\chi$ of $P$ such that $\widehat\mu(p)v=\chi(p)v$ for every $p\in P$. In particular, we would have for $u\in U$ that $\widehat\mu(u)v=v$ since $U=P'$. It follows that $v\in J\left(V_{\widehat\mu}\right)$; hence $v=0$, which is a contradiction. Therefore, by \Cref{thm:reps-of-p}, $V$ cannot have irreducible $P$-subspaces of dimension less than $q-1$. Hence, $V$ is irreducible and isomorphic to the unique $P$-representation $V_\pi$ of dimension $q-1$.
\end{proof}
\begin{lemma} \label{lem:mu-hat-red}
	If $\mu$ is a character of $B$ and $\widehat\mu$ is reducible, then the following are true.
	\begin{listalph}
		\item $\widehat\mu$ has a one-dimensional component.
		\item $\mu=\mu_w$.
	\end{listalph}
\end{lemma}
\begin{proof}
	Here we go.
	\begin{listalph}
		\item Let $V_{\widehat\mu}=V\oplus V'$ be a nontrivial $G$-decomposition of $V$. By \Cref{lem:decomp-res-mu-hat}, we can assume, without loss of generality, that $V_\pi\cap V\ne0$. Then $V_\pi\subseteq V$ because $V_\pi$ is an irreducible $P$-representation. On the other hand, by \Cref{lem:use-of-jac}, $J(V)\subseteq J(V_{\widehat\mu})\cap V$ is nonempty, so $V$ is not an irreducible $P$-representation. Hence, $V_\pi\ne V$. It follows that $\dim V=q$ and hence $\dim V'=1$.
		\item We have proved that there exists a character $\chi$ of $G$ and a nonzero function $f\colon G\to\CC$ such that
		\[f(bg)=\mu(b)f(g)\qquad\text{and}\qquad f(gs)=\chi(s)f(g)\]
		for every $b\in B$ and $g,s\in G$.
		
		We claim $f(1)\ne0$. Indeed, assume that $f(1)=0$ and let $g\in G$. Then there exists a positive integer $n$ such that $g^n=1$. Hence,
		\[0=f(1)=f\left(g\cdot g^{n-1}\right)=\chi\left(g^{n-1}\right)f(g).\]
		It follows that $f(g)=0$ because $\chi\left(g^{n-1}\right)\ne0$. This is a contradiction.

		Now, let $d\in D$; then $\mu(d)f(q)=f(d)=\chi(d)f(1)$. Hence, $\mu(d)=\chi(d)$. It follows that
		\[\mu_w(d)=\mu(wdw)=\chi(wdw)=\chi\left(w^2\right)\chi(d)=\mu(d).\]
		Hence, $\mu_w=\mu$.
		\qedhere
	\end{listalph}
\end{proof}
\begin{lemma} \label{lem:mu-hat-interference}
	Let $\mu$ and $\mu'$ be two distinct characters of $B$. Then $\left(\widehat\mu,\widehat\mu'\right)\ne0$ if and only if $\mu'=\mu_w$.
\end{lemma}
\begin{proof}
	Suppose first that $\left(\widehat\mu,\widehat\mu'\right)\ne0$. Then there exists an irreducible representation $\rho$ of $G$ such that $\rho\le\widehat\mu$ and $\rho\le\widehat\mu'$. It follows from \Cref{cor:irred-use-of-jac} that $J(V_\rho)\ne0$. By the proof of \Cref{lem:use-of-jac}, we know that $B$ operating on $J(V_\rho)$ has an eigenvalue $\chi$. In addition, by \Cref{lem:eigens-of-jac-mu-hat}, we know that $B$ operating on $J(V_{\widehat\mu})$ has exactly two eigenvalues $\mu$ and $\mu_w$. It follows that $\chi=\mu$ or $\chi=\mu_w$ because $J(V_\rho)\subseteq J(V_{\widehat\mu})$. Similarly, $\chi=\mu'$ or $\chi=\mu'_w$, but $\mu\ne\mu'$; hence, $\mu=\mu_w$.

	Conversely, suppose that $\mu'=\mu_w$. Then arguing backwards we have that $\left(\mu',\op{Res}^G_B\widehat\mu\right)\ne0$. Hence, by the Frobenius reciprocity theorem $\left(\widehat\mu',\widehat\mu\right)\ne0$.
\end{proof}
\begin{lemma} \label{lem:mu-hat-interference-full}
	Let $\mu$ and $\mu'$ be two distinct characters of $B$. Then $\widehat\mu=\widehat\mu'$ if and only if $\mu'=\mu_w$.
\end{lemma}
\begin{proof}
	Suppose that $\mu'=\mu_w$. Then $\mu\ne\mu_w$ and $\mu'=\mu'_w$. It follows that $\widehat\mu$ and $\widehat\mu'$ are irreducible by \Cref{lem:mu-hat-red}. Also, $\left(\widehat\mu,\widehat\mu'\right)\ne0$ by \Cref{lem:mu-hat-interference}. Hence, $\widehat\mu=\widehat\mu'$.

	The other direction of the lemma follows directly from \Cref{lem:mu-hat-interference}.
\end{proof}
Summing up the lemmas in this section, we obtain the theorem.
\begin{theorem} \label{thm:fully-decompose-mu-hat}
	Let $\mu$ and $\mu'$ be characters of $B$ and let $\widehat\mu\coloneqq\op{Ind}^G_B\mu$ and $\widehat\mu'\coloneqq\op{Ind}^G_B\mu'$.
	\begin{listalph}
		\item $\dim\widehat\mu=q+1$.
		\item $\widehat\mu$ has at most two irreducible components.
		\item $\widehat\mu$ is irreducible if and only if $\mu\ne\mu_w$.
		\item If $\widehat\mu$ is reducible, it decomposes into a direct sum of a one-dimensional and $q$-dimensional representation.
		\item $\widehat\mu'=\widehat\mu$ if and only if $\mu'=\mu$ or $\mu'=\mu_w$.
	\end{listalph}
\end{theorem}
\begin{proof}
	Note (a) is clear. \Cref{cor:mu-hat-decomp} gives (b). \Cref{lem:to-reducible,lem:mu-hat-red} give (c). \Cref{lem:to-reducible} and (b) give (d). Lastly, \Cref{lem:mu-hat-interference-full} gives (e).
\end{proof}

Now, consider a character $\mu$ of $B$ that corresponds to the pair of characters $(\mu_1,\mu_2)$ of $K^\times$. There are two possibilities.
\begin{listalph}
	\item $\mu=\mu_w$; i.e., $\mu_1=\mu_2$. In this case, $\widehat\mu$ is the direct sum of a one-dimensional representation $\rho'_{(\mu_1,\mu_2)}$ and an irreducible $q$-dimensional representation $\rho_{(\mu_1,\mu_2)}$. $K^\times$ has $q-1$ characters; hence, in this way we obtain $q-1$ characters of $G$ and $q-1$ different $q$-dimensional irreducible representations of $G$.
	\item $\mu\ne\mu_w$; i.e., $\mu_1\ne\mu_2$. In this case, $\widehat\mu$ is an irreducible representation of dimension $q+1$, and we denote it by $\rho_{(\mu_1,\mu_2)}$. The number of these $\mu$ is equal to the number of characters of $B$, i.e., $(q-1)^2$, minus the characters of type (a), i.e. $q-1$. Further, $\mu$ and $\mu_w$ induce the same representation. Hence, in this way we obtain $\frac12(q-1)(q-2)$ irreducible representations of $G$ of dimension $q-1$.
\end{listalph}
We have therefore proved the following theorem.
\begin{theorem} \label{thm:non-cusp-reps}
	The irreducible representations of $G$, which are components of induced representations of the form $\op{Ind}^G_B\mu$ where $\mu$ is a character of $B$, split up in the following classes.
	\begin{listalph}
		\item $q-1$ different $1$-dimensional representations $\rho'_{(\mu_1,\mu_1)}$.
		\item $q-1$ different $q$-dimensional representations $\rho_{(\mu_1,\mu_2)}$.
		\item $\frac12(q-1)(q-2)$ distinct $(q+1)$-dimensional representations, $\rho_{(\mu_1,\mu_2)}$.
	\end{listalph}
\end{theorem}
If $\chi$ is a character of $G$, then $\chi$ is a component of $\op{Ind}^G_B\op{Res}^G_B\chi$. It follows by \Cref{thm:non-cusp-reps} that $G$ has exactly $q-1$ characters. Hence, $[G:G^c]=q-1$. The subgroup $\op{SL}(2,K)=\{g\in G:\deg g=1\}$ is normal and $G/\op{SL}(2,K)\cong K^\times$. Hence, we have the following corollary.
\begin{corollary}
	$\op{SL}(2,K)$ is the commutator subgroup of $\op{GL}(2,K)$.
\end{corollary}

\section{The Schur algebra of \texorpdfstring{$\op{Ind}^G_B\mu$}{ Ind mu}}
Let $\mu$ be a character of $B$, and let $\widehat\mu\coloneqq\op{Ind}^G_B\mu$. Bruhat's decomposition of $G$ implies that $\left|B\backslash G/B\right|=2$. Hence, by \Cref{cor:dim-homs-of-ind}, $\left(\widehat\mu,\widehat\mu\right)\le2$. Because $\left(\widehat\mu,\widehat\mu\right)\ge1$, there are only two possibilities: either $\left(\widehat\mu,\widehat\mu\right)=1$, in which case $\widehat\mu$ is irreducible; or $\left(\widehat\mu,\widehat\mu\right)=2$. We also know that if $\widehat\mu=\bigoplus_{i=1}^rn_ip_i$ is the canonical decomposition of $\widehat\mu$, then
\[\left(\widehat\mu,\widehat\mu\right)=\sum_{i=1}^rn_i^2.\]
Because $2$ can be decomposed into a sum of squares only in the form $2=1^2+1^2$, it follows that $\left(\widehat\mu,\widehat\mu\right)=2$ implies that $\widehat\mu$ decomposes into a direct sum of two non-isomorphic representations. We have therefore proved the following theorem.
\begin{theorem} \label{thm:mu-hat-decomposition}
	Let $\mu$ be a character of $B$ and let $\widehat\mu\coloneqq\op{Ind}^G_B\mu$. Then either $\left(\widehat\mu,\widehat\mu\right)=1$, in which case $\widehat\mu$ is irreducible, or $\left(\widehat\mu,\widehat\mu\right)=2$, in which case $\widehat\mu$ decomposes into a direct sum of two non-isomorphic representations.
\end{theorem}
Obviously, \Cref{thm:mu-hat-decomposition} is also a consequence of \Cref{thm:fully-decompose-mu-hat}. However, the proof given above is independent of \Cref{thm:fully-decompose-mu-hat}.

\section{The dimension of cuspidal representations}
Irreducible representations of $G$ that are not components of $\widehat\mu$, with $\mu$ a character of $B$, are said to be \textit{cuspidal}. By \Cref{cor:irred-use-of-jac}, an irreducible representation $\rho$ of $G$ is cuspidal if and only if $J(V_\rho)=0$. Comparing \Cref{prop:conj-classes-of-g} with \Cref{thm:non-cusp-reps}, we find that $G$ has $\frac12\left(q^2-q\right)$ cuspidal representations, exactly the same as the number of conjugacy classes of the form $c_4(\alpha)$.

We delay a further explanation of this phenomenon to \cref{sec:weil-group} and concentrate in this section on proving that all cuspidal representations have dimension $q-1$. The first toward this goal is as follows.
\begin{lemma} \label{lem:almost-cuspidal-dim}
	Let $\rho$ be a cuspidal representation of $G$. Then $\op{Res}^G_P\rho=r\pi$ for some positive integer $r$. In particular, $\dim\rho=r(q-1)$ is a multiple of $q-1$.
\end{lemma}
\begin{proof}
	The $P$-representation $\op{Res}^G_PV_\rho$ cannot have one-dimensional components. Indeed, otherwise, there would exist a nonzero vector $v\in V_\rho$ and a character $\chi$ of $P$ such that $\rho(p)v=\chi(p)v$ for every $p\in P$. In particular, we would have $\rho(u)v=v$ for every $u\in U$; i.e., $v\in J(V_\rho)$; thus, $J(V_\rho)\ne0$, contrary to the assumption that $\rho$ is cuspidal (see also the proof of \Cref{lem:decomp-res-mu-hat}). By \Cref{thm:reps-of-p}, $\op{Res}^G_P\rho$ must be a multiple of $\pi$.
\end{proof}
\begin{proposition} \label{prop:cuspidal-dim}
	The following are true.
	\begin{listalph}
		\item Let $\rho$ be a cuspidal representation. Then $\op{Res}^G_P\rho=\pi$ and $\dim\rho=q+1$.
		\item Conversely, if $\rho$ is a representation of $G$ such that $\op{Res}^G_P\rho=\pi$, then $\rho$ is cuspidal.
	\end{listalph}
\end{proposition}
\begin{proof}
	Here we go.
	\begin{listalph}
		\item Using the formula $\left|G\right|=\sum_\sigma(\dim\sigma)^2$ where $\sigma$ runs over the irreducible representations of $G$, and by \Cref{thm:non-cusp-reps}, we have
		\[(q-1)^2q(q+1)\ge(q-1)\cdot1^2+(q-1)\cdot q^2+\frac12(q-1)(q-2)\cdot(q+1)^2+\sum_{\sigma\text{ cuspidal}}(\dim\sigma)^2.\]
		By \Cref{lem:almost-cuspidal-dim}, there exists for every $\sigma$ a positive integer $r(\sigma)$ such that $\dim\sigma=(q-1)r(\sigma)$. Hence,
		\[\frac12\left(q^2-q\right)\ge\sum_{\sigma\text{ cuspidal}}r(\sigma).\]
		The number of the summands on the right-hand side is equal to $\frac12\left(q^2-q\right)$. Hence, $r(\sigma)=1$, and (a) follows from \Cref{lem:almost-cuspidal-dim}.
		\item The representation $\pi$ is irreducible; hence $\rho$ is irreducible too. Also, if $\mu$ is a character of $B$, then by \Cref{thm:non-cusp-reps}, the components of $\widehat\mu$ have the dimensions $1$, $q$, or $q=1$. However, $\dim\rho=\dim\pi=q-1$; hence, $\rho$ is not equal to any of them (i.e., $\rho$ is cuspidal).
		\qedhere
	\end{listalph}
\end{proof}
The proof of \Cref{prop:cuspidal-dim}(a) relies heavily on the fact that $K$ is a finite field. We now give another proof that will be independent of this fact.

Let $\widehat\pi\coloneqq\op{Ind}^G_B=\op{Ind}^G_U\psi$. Then by \Cref{prop:homs-of-ind}, $\op{End}_{\CC[G]}V_{\widehat\pi}$ is isomorphic to the algebra $A$ of all functions $F\colon G\to\CC$ satisfying
\[F(u_1gu_2)=\psi(u_1u_2)F(g)\qquad\text{for }u_1,u_2\in U\text{ and }g\in G,\]
where multiplication between two functions $F_1,F_2\in A$ is given by the formula
\[(F_1*F_2)(g)=\frac1{[G:U]}\sum_{s\in G}F_1\left(gs^{-1}\right)F_2(s)\]
(see \eqref{eq:desired-schur-iso}). We will show that $A$ is abelian. This implies that $\widehat\pi$ has no multiple components (see \cref{sec:intro-line-reps}). If $\rho$ is a cuspidal representation, then by \Cref{lem:almost-cuspidal-dim} there exists a positive integer $r$ such that $\left(\op{Res}^G_P\rho,\pi\right)=r$ and $\dim\rho=r(q-1)$. Hence, by the Frobenius reciprocity theorem, $r=(\rho,\widehat\pi)=1$, and our contention is proved.

Our method of proving that $A$ is abelian is indirect. We shall define an involution on $A$; i.e., a map $F\mapsto F'$ such that $(F_1*F_2)'=F_2'*F_1'$. We further prove that $F=F'$ for every $F\in A$. Hence, $F_1*F_2=F_2*F_1$.

We start by defining an involution $g\mapsto g'$ on $G$: if $g=\begin{bmatrix}
	\alpha & \beta \\
	\gamma & \delta
\end{bmatrix}$, we let $g'\coloneqq\begin{bmatrix}
	\delta & \beta \\
	\gamma & \alpha
\end{bmatrix}$. Then $(g_1g_2)'=g_2'g_1'$, $g''=g$, and $g=g'$ if $g$ is symmetric with respect to the second diagonal. In particular, $u'=u$ for every $u\in U$. We continue by defining for an element $F\in A$ a function $F'\colon G\to\CC$ by $F'(g)\coloneqq F(g')$. Then $F'$ belongs to $A$. In order prove that $F=F'$, it suffices to show that $F$ and $F'$ coincide on representatives of the double classes $U\backslash G/U$. Indeed, by Bruhat's decomposition $G=B\sqcup BwU$ and because $B=UD$, we have that the above representations are either of the form
\[\text{(a) }\begin{bmatrix}
	0 & \beta \\
	\gamma & 0
\end{bmatrix},\qquad\text{or of the form (b) }\begin{bmatrix}
	\alpha & 0 \\
	0 & \delta
\end{bmatrix}.\]
Clearly, $F$ and $F'$ coincide on the matrices (a) and on the matrices (b) in case $\alpha=\delta$. In order to prove that $F$ and $F'$ coincide also on the matrices (b) in case $\alpha\ne\delta$, it suffices to show that $F$ (and hence also $F'$) vanishes on them. Indeed, acting with $F$ on both sides of the identity
\[\begin{bmatrix}
	1 & \beta \\
	0 & 1
\end{bmatrix}\begin{bmatrix}
	\alpha & 0 \\
	0 & \delta
\end{bmatrix}=\begin{bmatrix}
	\alpha & 0 \\
	0 & \delta
\end{bmatrix}\begin{bmatrix}
	1 & \alpha^{-1}\delta\beta \\
	0 & 1
\end{bmatrix},\]
we have
\[\psi(\beta)F\left(\begin{bmatrix}
	\alpha & 0 \\
	0 & \delta
\end{bmatrix}\right)=F\left(\begin{bmatrix}
	\alpha & 0 \\
	0 & \delta
\end{bmatrix}\right)\psi\left(\alpha^{-1}\delta\beta\right).\]
If $F\left(\begin{bmatrix}
	\alpha & 0 \\
	0 & \delta
\end{bmatrix}\right)\ne0$, then $\psi(\beta)=\psi\left(\alpha^{-1}\delta\beta\right)$; hence, $\psi\left(\beta\left(1-\alpha^{-1}\delta\right)\right)=1$ for every $\beta\in K$. It follows that $\psi$ is the unit character of $K^+$, which is a contradiction.

This completes the alternative proof of \Cref{prop:cuspidal-dim}(a). Note that we have actually also proved the following proposition.
\begin{proposition} \label{prop:ind-psi-has-no-mult}
	The representation $\op{Ind}^G_U\psi$ has no multiple components.
\end{proposition}

\section{The description of \texorpdfstring{$\GL(2,K)$}{ GL(2,K)} by generators and relations}
We need this for an explicit description of the cuspidal representations. Let
\[w'\coloneqq\begin{bmatrix}
	0 & 1 \\
	-1 & 0
\end{bmatrix},\qquad z\coloneqq\begin{bmatrix}
	1 & 1 \\
	0 & 1
\end{bmatrix},\qquad\text{and}\qquad s\coloneqq w'z.\]
Then we have the following relations between $w'$ and the elements of $B$:
\begin{align}
	w'\begin{bmatrix}
		\alpha & 0 \\
		0 & \delta
	\end{bmatrix}(w')^{-1} &= \begin{bmatrix}
		\beta & 0 \\
		0 & \alpha
	\end{bmatrix}, \label{eq:wp-diag-relation} \\
	(w')^2 &= \begin{bmatrix}
		-1 & 0 \\
		0 & -1
	\end{bmatrix} \label{eq:wp2-relation} \\
	s^3 &= 1. \label{eq:s3-relation}
\end{align}
\begin{proposition} \label{prop:gl-by-relations}
	The group $\op{GL}(2,K)$ is the free group generated by $B$ and $w'$ with \eqref{eq:wp-diag-relation}, \eqref{eq:wp2-relation}, and \eqref{eq:s3-relation} as the defining relations.
\end{proposition}
\begin{proof}
	Denote by $G$ the free group generated by $B$ and $w'$ with the above defining relations. Then there exists a unique epimorphism $\theta$ of $G$ onto $\op{GL}(2,K)$ which is the identity on $B$ and maps $w'$ onto itself. We have to prove that its kernel consists of $1$.

	We claim that for every $b\in B\setminus D$, there exists $b_1,b_2\in B$ such that $w'bw'=b_1w'b_2$. Indeed, if
	\[b\coloneqq\begin{bmatrix}
		\alpha & \beta \\
		0 & \delta
	\end{bmatrix}\]
	and $\beta\ne0$, then
	\[b=\begin{bmatrix}
		1 & 0 \\
		0 & \delta\beta^{-1}
	\end{bmatrix}\begin{bmatrix}
		1 & 1 \\
		0 & 1
	\end{bmatrix}\begin{bmatrix}
		\alpha & 0 \\
		0 & \delta
	\end{bmatrix}=d'zd'';\]
	also $w'zw'zw'z=1$ by \eqref{eq:s3-relation}; hence,
	\[w'zw'=z^{-1}(w')^{-1}z^{-1}=z^{-1}\begin{bmatrix}
		-1 & 0\\
		0 & -1
	\end{bmatrix}w'z^{-1}\]
	by \eqref{eq:wp2-relation}. It follows that
	\[w'bw'=\left(w'd'(w')^{-1}\right)w'zw'\left((w')^{-1}d''w'\right)=\left(w'd'(w')^{-1}\right)z^{-1}\begin{bmatrix}
		-1 & 0 \\
		0 & -1
	\end{bmatrix}w'z^{-1}\left((w')^{-1}d''w'\right)=b_1w'b_2,\]
	by \eqref{eq:wp-diag-relation}. This completes the proof of the claim.

	Next, note that if $d\in D$, then $w'dw'=\left(w'd(w')^{-1}\right)(w')^2\in B$ by \eqref{eq:wp-diag-relation} and \eqref{eq:wp2-relation}. Now, let $g\ne1$ be in the kernel of $\theta$. Then $g\notin B$; hence, $g$ can be written, for example, as $g=b_0w'b_1\cdots w'b_r$, where $b_i\in B$. If $r\ge2$, then either $w'b_1w'=b_1'\in B$, if $b_1\in D$, or $w'b_1w'=b_1'w'b_2'$ with $b_1',b_2'\in D$ if $b_1\in B\setminus D$. In any case one can rewrite $g$ as $g=b_2''w'\ldots w'b_r$. By a repeated application of this procedure, one finally proves that $g\in Bw'B$; i.e.,
	\[g=\begin{bmatrix}
		\alpha' & \beta' \\
		0 & \delta'
	\end{bmatrix}w'\begin{bmatrix}
		\alpha & \beta \\
		0 & \delta
	\end{bmatrix}.\]
	The right-hand side is mapped by $\theta$ to an element
	\[\begin{bmatrix}
		* & * \\
		-\delta'\alpha & *
	\end{bmatrix}\]
	of $\op{GL}(2,K)$. But $g$ is mapped to $1$. Hence, $\delta'\alpha=0$, which is a contradiction.
\end{proof}

\section{Non-decomposable characters of \texorpdfstring{$L^\times$}{ L*}}
We have denoted by $L$ the unique quadratic extension of $K$. It has $q^2$ elements. If $\alpha$ is an element of $L$, then $\overline\alpha$ denotes its unique conjugate over $K$. Then the function $\op N\alpha\coloneqq\alpha\overline\alpha$ is the norm map from $L$ to $K$. It is multiplicative.
\begin{lemma} \label{lem:n-epi}
	The function $\op N$ is surjective.
\end{lemma}
\begin{proof}
	The Galois group of $L$ over $K$ is generated by the Frobenius automorphism $\alpha\mapsto\alpha^q$. Hence, $\op N\alpha=\alpha^{q+1}$. The restriction of $\op N$ to $L^\times$ is a homomorphism to $K^\times$. It follows that the kernel $E$ of this homomorphism consists of $q+1$ elements. Hence, its image consists of $\frac{q^2-1}{q+1}=q-1$ elements, exactly as many elements as $K^\times$ has.
\end{proof}

We note that \Cref{lem:n-epi} also follows from the fact that the Brauer group of $K$ is trivial. The proof of the lemma is not so elementary as the one we gave, but it is independent of counting elements.
\begin{corollary}[Hilbert's Satz 90] \label{cor:h90}
	If $\beta$ is an element of $L^\times$ such that $\op N\beta=1$, then there exists an $\alpha\in L$ such that $\alpha\overline\alpha^{-1}=\beta$.
\end{corollary}
\begin{proof}
	Let $E\coloneqq\ker\op N$. The map $h\colon L^\times\to E$ defined by $h(\alpha)\coloneqq\alpha\overline\alpha^{-1}$ is a homomorphism. Its kernel is $K^\times$. Hence, the image of $h$ as $\left(q^2-1\right)(q-1)^{-1}$ elements, exactly as many as $E$ has.
\end{proof}
\begin{definition}[decomposable]
	Let $\chi$ be a character of $K^\times$. Composing $\chi$ with the norm map $\op N$ from $L^\times$ to $K^\times$, we obtain a character $\widetilde\chi$ of $L^\times$ by
	\begin{equation}
		\widetilde\chi(\alpha)\coloneqq\chi(\op N\alpha) \qquad\text{for }\alpha\in L^\times. \label{eq:define-decomp-char}
	\end{equation}
	Here, $\widetilde\chi$ is said to be \textit{decomposable}.
\end{definition}
\begin{notation}
	If $\nu$ is an arbitrary character of $L^\times$, then $\overline\nu$ denotes its conjugate over $K$; i.e., $\overline\nu(\alpha)\coloneqq\nu(\overline\alpha)$ for $\alpha\in L^\times$.
\end{notation}
\begin{lemma} \label{lem:decomp-criterion}
	A character $\nu$ of $L^\times$ is decomposable if and only if $\nu=\overline\nu$.
\end{lemma}
\begin{proof}
	If there exists a character $\chi$ of $K^\times$ such that $\nu=\widetilde\chi$, then certainly $\nu(\alpha)=\nu(\overline\alpha)$ for every $\alpha\in K^\times$. Conversely, if $\nu=\overline\nu$, then we define $\chi(\op N\alpha)\coloneqq\nu(\alpha)$ for $\alpha\in L^\times$. If $\beta\in L^\times$ is such that $\op N\alpha=\op N\beta$, then by \Cref{cor:h90}, there exists a $\gamma\in L^\times$ such that $\alpha\beta^{-1}=\gamma\overline\gamma^{-1}$; hence, $\nu(\alpha)=\nu(\beta)$, and therefore $\chi(\op N\alpha)$ is well-defined. The fact that $\op N$ is surjective now extends the domain of definition of $\chi$ to $K^\times$. Hence, $\chi$ is a character of $K^\times$ and $\nu$ is therefore decomposable.
\end{proof}
\begin{lemma} \label{lem:sum-over-equal-norm}
	If $\nu$ is a non-decomposable character of $L^\times$, then
	\[\sum_{\op Nx=\alpha}\nu(x)=0\]
	for every $\alpha\in K^\times$.
\end{lemma}
\begin{proof}
	By the proof of \Cref{lem:decomp-criterion}, there exists a $\lambda\in L^\times$ such that $\op N\lambda=1$ and $\nu(\lambda)\ne1$. Hence,
	\[\sum_{\op Nx=\alpha}\nu(x)=\sum_{\op Nx=\alpha}\nu(\lambda x)=\nu(\lambda)\sum_{\op Nx=\alpha}\nu(x),\]
	and our claim follows.
\end{proof}

We shall need the analogue to \Cref{lem:n-epi} for the trace function $\op{Tr}\colon L^+\to K^+$ defined by $\op{Tr}x\coloneqq x+\overline x$.
\begin{lemma} \label{lem:tr-epi}
	The function $\op{Tr}$ is surjective.
\end{lemma}
\begin{proof}
	The trace function is a homomorphism. Its kernel consists of those $x$ in $L$ that satisfy $x+x^q=0$. It contains therefore $q$ elements. Therefore, $\op{Tr}$ is surjective.
\end{proof}
\begin{corollary} \label{cor:use-tr-epi}
	If $\alpha\in L^\times$, then for every $\beta\in K$, there exists an $x\in L$ such that $\alpha x+\overline\alpha\overline x=\beta$.
\end{corollary}
\begin{proof}
	This follows from \Cref{lem:tr-epi}.
\end{proof}

\section{Assigning cuspidal reps. to non-decomposable characters} \label{sec:def-cusp}
Let $\nu$ be a non-decomposable character. We are going to define a representation $\rho=\rho_\nu$ of $G$ that will turn out to be a cuspidal representation. In order to define $\rho$ on $G$, it suffices by \Cref{prop:gl-by-relations} to define $\rho$ as a map from $B\cup\{w'\}$ into the automorphism group of an appropriate vector space $B$ such that the restriction of $\rho$ to $B$ is a homomorphism and such that $\rho$ preserves the relation \eqref{eq:wp-diag-relation}, \eqref{eq:wp2-relation}, and \eqref{eq:s3-relation}. The dimension of $\rho$ should be $q-1$ by \Cref{prop:cuspidal-dim}. Hence, it is convenient to take $V$ as the vector space of all functions $K^\times\to\CC$.

The definition of $\op{Res}^G_P\rho$ is motivated by the fact proved in \Cref{prop:cuspidal-dim} that it should be equal to $\pi$. Identifying the subgroup $A$ of $P$ with $K^\times$ and using the identity
\[\begin{bmatrix}
	1 & \beta \\
	0 & 1
\end{bmatrix}\begin{bmatrix}
	\alpha & 0 \\
	0 & 1
\end{bmatrix}=\begin{bmatrix}
	\alpha & \beta \\
	0 & 1
\end{bmatrix},\]
we are led to define
\[\left(\rho\left(\begin{bmatrix}
	\alpha & \beta \\
	0 & 1
\end{bmatrix}\right)f\right)(x)=\psi(\beta x)f(\alpha x).\]
Further, we would like to define that $\rho$ coincides with $\nu$ on $D$:
\[\left(\rho\left(\begin{bmatrix}
	\delta & 0 \\
	0 & \delta
\end{bmatrix}\right)f\right)(x)=\nu(\delta)f(x).\]
It follows that we must define $\rho$ on $B$ by
\begin{equation}
	\left(\rho\left(\begin{bmatrix}
		\alpha & \beta \\
		0 & \delta
	\end{bmatrix}\right)f\right)(x)\coloneqq\nu(\delta)\psi\left(\beta\delta^{-1}x\right)f\left(\alpha\delta^{-1}x\right). \label{eq:def-cusp-b}
\end{equation}
A straightforward calculation shows that $\rho$ is indeed a homomorphism of $B$ into $\op{Aut}V$.

In order to define $\rho(w')$, we define a function $j\colon K^\times\to\CC$ by
\begin{equation}
	j(u)\coloneqq\frac1q\sum_{\substack{\op Nt=u\\t\in L^\times}}\psi\left(t+\overline t\right)\nu(t) \label{eq:def-cusp-j}
\end{equation}
and prove that it satisfies the following two identities.
\begin{lemma} \label{lem:j1}
	For $u\in K^\times$,
	\[\sum_{v\in K^\times}j(uv)j(v)\nu\left(v^{-1}\right)=\begin{cases}
		\nu(-1) & \text{if }u=1, \\
		0 & \text{if }u\ne1.
	\end{cases}\]
\end{lemma}
\begin{proof}
	We start from the left-hand side. Letting $L(u)$ denote the left-hand side,
	\begin{align}
		L(u) &= \sum_{v\in K^\times}j(uv)j(v)\nu\left(v^{-1}\right) \notag \\
		&= \sum_{v\in K^\times}q^{-2}\sum_{\substack{\op Nt=uv\\\op Ns=v}}\psi\left(t+\overline t+s+\overline s\right)\nu(ts)\nu\left(v^{-1}\right) \notag \\
		&= q^{-2}\sum_{v\in K^\times}\sum_{\substack{\op Nt=u\op Ns\\\op Ns=v}}\psi\left(t+\overline t+s+\overline s\right)\nu\left(ts\op Ns^{-1}\right) \notag \\
		&= q^{-2}\sum_{s\in L^\times}\sum_{\op Nt=u\op Ns}\psi\left(t+\overline t+s+\overline s\right)\nu\left(t\overline s^{-1}\right). \label{eq:j1-lhs1}
	\end{align}
	Let $\lambda\coloneqq t\overline s^{-1}$. Then substituting $t=\overline s\lambda$ in \eqref{eq:j1-lhs1}, we have
	\begin{align}
		L(u) &= q^{-2}\sum_{s\in L^\times}\sum_{\op N\lambda=u}\psi\left(s(1+\lambda)+\overline s(1+\overline\lambda)\right)\nu(\lambda) \notag \\
		&= q^{-2}\sum_{\op N\lambda=u}\nu(\lambda)\sum_{s\in L^\times}\psi\left(s(1+\lambda)+\overline s(1+\overline\lambda)\right). \label{eq:j1-lhs2}
	\end{align}
	For a fixed $\lambda$, the function $\psi\left(s(1+\lambda)+\overline s(1+\overline\lambda)\right)$ is a character of $L^+$. If $\lambda\ne-1$, then this is not the unit character, because by \Cref{cor:use-tr-epi} the map $s\mapsto s(1+\lambda)+\overline s(1+\overline\lambda)$ maps $L$ onto $K$ and $\psi$ is not the unit character. It follows that
	\[\sum_{s\in L^\times}\psi\left(s(1+\lambda)+\overline s(1+\overline\lambda)\right)=-1.\]
	If $\lambda=-1$, we have
	\[\sum_{s\in L^\times}\psi\left(s(1+\lambda)+\overline s(1+\overline\lambda)\right)=q^2-1.\]
	We now distinguish between the two cases and suppose first that $u=1$. Then \eqref{eq:j1-lhs2} is equal to
	\begin{align}
		L(u) &= q^{-2}\sum_{\substack{\op N\lambda=1\\\lambda\ne-1}}\nu(\lambda)\sum_{s\in L^\times}\psi\left(s(1+\lambda)+\overline s(1+\overline\lambda)\right)+q^{-2}\nu(-1)\left(q^2-1\right) \\
		&= q^{-2}\sum_{\substack{\op N\lambda=1\\\lambda\ne-1}}(-\nu(\lambda))+\nu(-1)q^{-2}\left(q^2+1\right).
	\end{align}
	Using \Cref{lem:sum-over-equal-norm}, we may continue this chain of equalities by
	\[L(u)=q^{-2}\nu(-1)+\nu(-1)q^{-2}\left(q^2+1\right)=\nu(-1),\]
	as desired.

	Now suppose that $u\ne1$. Then $\op N\lambda=u$ implies $\lambda\ne-1$. Hence, \eqref{eq:j1-lhs2} is equal in this case to
	\[q^{-2}\sum_{\op N\lambda=u}(-\nu(\lambda))=0\]
	by \Cref{lem:sum-over-equal-norm}, as desired.
\end{proof}
\begin{lemma} \label{lem:j2}
	For $x,y\in K^\times$,
	\[\sum_{v\in K^\times}j(xv)j(yv)\nu\left(v^{-1}\right)\psi(v)=\nu(-1)\psi(-x-y)j(xy).\]
\end{lemma}
\begin{proof}
	We start again from the left-hand side. Let $x,y\in K^\times$; then letting $L(x,y)$ denote the left-hand side,
	\begin{align}
		L(x,y) &= \sum_{v\in K^\times}j(xv)j(yv)\nu\left(v^{-1}\right)\psi(v) \notag \\
		 &= q^{-2}\sum_{v\in K^\times}\sum_{\substack{\op Nt=xv\\\op Ns=yv}}\psi\left(t+\overline t+s+\overline s\right)\nu(ts)\nu\left(v^{-1}\right)\psi(v) \notag \\
		&= q^{-2}\sum_{v\in K^\times}\sum_{\substack{\op Nt=xv\\\op Ns=yv}}\psi\left(t+\overline t+s+\overline s+v\right)\nu\left(tsv^{-1}\right). \label{eq:j2-lhs1}
	\end{align}
	The condition $\op Ns=yv$ implies that $tsv^{-1}=yt\overline s^{-1}$. Define therefore $\lambda\coloneqq yt\overline s^{-1}$. In addition, $\op Nt=xv$ implies $\op N\lambda=xy$. Also,
	\[t+\overline t+s+\overline s+v=y^{-1}(s+y+\lambda)\overline{(s+y+\lambda)}-y\left(1+y^{-1}\lambda\right)\overline{\left(1+y^{-1}\lambda\right)}.\]
	Substituting this all in \eqref{eq:j2-lhs1} gives
	\begin{align}
		L(x,y) &= q^{-2}\sum_{v\in K^\times}\sum_{\substack{\op Ns=yv\\\op N\lambda=xy}}\psi\left(y^{-1}(s+y+\lambda)\overline{(s+y+\lambda)}-y\left(1+y^{-1}\lambda\right)\overline{\left(1+y^{-1}\lambda\right)}\right)\nu(\lambda) \notag \\
		&= q^{-2}\sum_{s\in L^\times}\sum_{\op N\lambda=xy}\psi\left(y^{-1}(s+y+\lambda)\overline{(s+y+\lambda)}-y\left(1+y^{-1}\lambda\right)\overline{\left(1+y^{-1}\lambda\right)}\right)\nu(\lambda) \notag \\
		&= q^{-2}\sum_{\op N\lambda=xy}\psi\left(-y\left(1+y^{-1}\lambda\right)\overline{\left(1+y^{-1}\lambda\right)}\right)\nu(\lambda)\sum_{s\in L^\times}\psi\left(y^{-1}(s+y+\lambda)\overline{(s+y+\lambda)}\right). \label{eq:j2-lhs2}
	\end{align}
	Let us develop the inner sums
	\begin{align*}
		\sum_{s\in L^\times}\psi\left(y^{-1}(s+y+\lambda)\overline{(s+y+\lambda)}\right) &= \sum_{\substack{r\in L\\r\ne y+\lambda}}\psi\left(y^{-1}r\overline r\right) \\
		&= \sum_{r\in L^\times}\psi\left(y^{-1}r\overline r\right)+1-\psi\left(y^{-1}(y+\lambda)\overline{(y+\lambda)}\right) \\
		&= (q+1)\sum_{u\in K^\times}\psi\left(y^{-1}u\right)+1-\psi\left(y^{-1}(y+\lambda)(y+\overline\lambda)\right) \\
		&= -(q+1)+1-\psi\left(y^{-1}(y+\lambda)(y+\overline\lambda)\right).
	\end{align*}
	We have used the fact that $\ker\op N$ consists of $q+1$ elements. Substituting this into \eqref{eq:j2-lhs2}, we obtain that \eqref{eq:j2-lhs2} is equal to
	\begin{align}
		L(x,y) &= q^{-2}\sum_{\op N\lambda=xy}\psi\left(-y\left(1+y^{-1}\lambda\right)\overline{\left(1+y^{-1}\lambda\right)}\right)\nu(\lambda)\left(-q-\psi\left(y^{-1}(y+\lambda)(y+\overline\lambda)\right)\right) \notag \\
		&= q^{-1}\sum_{\op N\lambda=xy}\psi\left(-y\left(1+y^{-1}\lambda\right)\overline{\left(1+y^{-1}\lambda\right)}\right)\nu(\lambda) \notag \\
		&\qquad-q^{-2}\sum_{\op N\lambda=xy}\psi\left(-y\left(1+y^{-1}\lambda\right)\overline{\left(1+y^{-1}\lambda\right)}+y^{-1}(y+\lambda)(y+\overline\lambda)\right). \label{eq:j2-lhs3}
	\end{align}
	In order to compute the two sums, note that under the assumption $\lambda\overline\lambda=xy$ we have
	\[-y\left(1+y^{-1}\lambda\right)\left(1+y^{-1}\overline\lambda\right)=-y-x-(\lambda+\overline\lambda),\]
	and
	\[-y\left(1+y^{-1}\lambda\right)\left(1+y^{-1}\overline\lambda\right)+y^{-1}(y+\lambda)(y+\overline\lambda)=0.\]
	Hence, \eqref{eq:j2-lhs3} is equal to
	\begin{align*}
		L(x,y) &= -q^{-1}\psi(-x-y)\sum_{\op N\lambda=xy}\psi\left(-\lambda-\overline\lambda\right)\nu(\lambda)-q^{-2}\sum_{\op N\lambda=xy}\nu(\lambda) \\
		&= -q^{-1}\psi(-x-y)\nu(-1)\sum_{\op N\lambda=xy}\psi\left(\lambda+\overline\lambda\right)\nu(\lambda) \\
		&= \nu(-1)\psi(-x-y)j(xy),
	\end{align*}
	as desired.
\end{proof}
Having proved the above identities, we consider an $f\in V$ and define $\rho(w')f$ by
\begin{equation}
	\left(\rho(w')f\right)(y) \coloneqq \sum_{x\in K^\times}\nu\left(x^{-1}\right)j(yx)f(x). \label{eq:def-cusp-wp}
\end{equation}
Our task now is to prove that this definition of $\rho(w')$ together with \eqref{eq:def-cusp-b} is compatible with the identities \eqref{eq:wp-diag-relation}, \eqref{eq:wp2-relation}, and \eqref{eq:s3-relation}.

Indeed, it is convenient to write \eqref{eq:wp-diag-relation} in the form
\[w'\begin{bmatrix}
	\alpha & 0 \\
	0 & \delta
\end{bmatrix}=\begin{bmatrix}
	\delta & 0 \\
	0 & \alpha
\end{bmatrix}w'.\]
A straightforward calculation shows that the two automorphisms obtained by acting with $\rho$ on both sides of the equation operate in the same way on every element $f\in V$.

In order to show that $\rho$ preserves identity \eqref{eq:wp2-relation}, we compute for an $f\in V$ that
\begin{align}
	(\rho(w')\rho(w')f)(z) &= \sum_{x\in K^\times}\nu\left(x^{-1}\right)j(zx)(\rho(w')f)(x) \notag \\
	&= \sum_{x\in K^\times}\nu\left(x^{-1}\right)j(zx)\sum_{y\in K^\times}\nu\left(y^{-1}\right)j(xy)f(x) \notag \\
	&= \sum_{y\in K^\times}\nu\left(y^{-1}\right)f(y)\sum_{x\in K^\times}j(xy)j(zx)\nu\left(x^{-1}\right). \label{eq:wp2-check}
\end{align}
Changing variables by $zx=c$ and $xy=uv$, \eqref{eq:wp2-check} is equal to
\[\sum_{u\in K^\times}\nu\left(y^{-1}\right)f(uz)\nu(z)\sum_{v\in K^\times}j(uv)j(v)\nu\left(v^{-1}\right)=\nu\left(z^{-1}\right)\nu(z)\nu(-1)f(z)=\left(\rho\left(\begin{bmatrix}
	-1 & 0 \\
	0 & -1
\end{bmatrix}\right)f\right)(z)\]
by \eqref{eq:def-cusp-b} and \Cref{lem:j1}.

Finally, we have to prove that $\rho$ preserves relation \eqref{eq:s3-relation}. This is done by rewriting it as
\[w'\begin{bmatrix}
	1 & 1 \\
	0 & 1
\end{bmatrix}w'=\begin{bmatrix}
	-1 & 1 \\
	0 & -1
\end{bmatrix}w'\begin{bmatrix}
	1 & -1 \\
	0 & 1
\end{bmatrix},\]
making the necessary computations as above, and using \Cref{lem:j2}. The actual computations raise no significant problem, so we omit them.

We have thus proved that starting from a non-decomposable character $\nu$ of $L^\times$, there exists a representation $\rho=\rho_\nu$ of $G$ that acts on $B$ via \eqref{eq:def-cusp-b} and on $w'$ via \eqref{eq:def-cusp-wp}. For later references, let us also describe the action of $\rho$ on the element
\begin{equation}
	g\coloneqq\begin{bmatrix}
		\alpha & \beta \\
		\gamma & \delta
	\end{bmatrix} \label{eq:g-notin-b}
\end{equation}
of $G$, where $\gamma\ne0$. We use the identity
\[\begin{bmatrix}
	\alpha & \beta \\
	\gamma & \delta
\end{bmatrix}=\begin{bmatrix}
	\beta-\alpha\gamma^{-1}\delta & -\alpha \\
	0 & -\gamma
\end{bmatrix}\begin{bmatrix}
	0 & -1 \\
	-1 & 0
\end{bmatrix}\begin{bmatrix}
	1 & \gamma^{-1}\delta \\
	0 & 1
\end{bmatrix}\]
and apply \eqref{eq:def-cusp-b} and \eqref{eq:def-cusp-wp} to compute the action of $\rho(g)$ on a function $f\colon K^\times\to\CC$ as
\begin{align*}
	(\rho(g)f)(y) &= \nu(-\gamma)\psi\left(\alpha\gamma^{-1}y\right)\left(\rho(w')\rho\left(\begin{bmatrix}
		1 & \gamma^{-1}\delta \\
		0 & 1
	\end{bmatrix}\right)f\right)\left(\left(\beta-\alpha\gamma^{-1}\delta\right)(-\gamma)^{-1}y\right) \\
	&= \nu(-\gamma)\psi\left(\alpha\gamma^{-1}y\right)\sum_{x\in K^\times}\nu\left(x^{-1}\right)j\left(\left(\alpha\gamma^{-1}\delta-\beta\right)\gamma^{-1}yx\right)\left(\rho\left(\begin{bmatrix}
		1 & \gamma^{-1}\delta \\
		0 & 1
	\end{bmatrix}\right)f\right)(x) \\
	&= \nu(-\gamma)\psi\left(\alpha\gamma^{-1}y\right)\sum_{x\in K^\times}\nu\left(x^{-1}\right)j\left(\gamma^{-2}yx\det g\right)\psi\left(\gamma^{-1}\delta x\right)f(x) \\
	&= \nu(-\gamma)\psi\left(\alpha\gamma^{-1}y\right)\sum_{x\in K^\times}\nu\left(x^{-1}\right)\psi\left(\gamma^{-1}\delta x\right)f(x)\left(q^{-1}\right)\sum_{\substack{t\overline t=\gamma^{-2}yx\det g\\t\in L^\times}}\psi\left(t+\overline t\right)\nu(t).
\end{align*}
Substituting $t=-\gamma^{-1}xu$,
\[(\rho(g)f)(y)=\sum_{x\in K^\times}\left[q^{-1}\psi\left(\frac{\alpha y+\delta x}\gamma\right)\sum_{\substack{u\overline u=yx^{-1}\det g\\u\in L^\times}}\psi\left(-\frac x\gamma(u+\overline u)\right)\nu(u)\right]f(x).\]
We have therefore proved that for an element $g$ given by \eqref{eq:g-notin-b},
\[(\rho(g)f)(y)=\sum_{x\in K^\times}k(y,x;g)f(x),\]
where
\[k(y,x;g)\coloneqq\frac1q\psi\left(\frac{\alpha y+\delta x}\gamma\right)\sum_{\substack{u\overline u=yx^{-1}\det g\\u\in L^\times}}\psi\left(-\frac x\gamma(u+\overline u)\right)\nu(u).\]

\section{The correspondence between \texorpdfstring{$\nu$}{ v} and \texorpdfstring{$\rho_\nu$}{ pv}} \label{sec:classify-cusp}
\begin{proposition} \label{prop:cusp-reps}
	The following are true.
	\begin{listalph}
		\item If $\nu$ is a non-decomposable character of $L^\times$, then the representation $\rho_\nu$ of $G$ defined in \cref{sec:def-cusp} is cuspidal.
		\item If $\nu$ and $\nu'$ are non-decomposable character of $L^\times$, then $\rho_\nu$ is isomorphic to $\rho_{\nu'}$ if and only if $\nu$ is conjugate to $\nu'$ over $K$.
	\end{listalph}
\end{proposition}
\begin{proof}
	Here we go.
	\begin{listalph}
		\item We have defined $\rho_\nu$ such that its restriction to $P$ is equal to $\pi$.\footnote{By definition, $\op{Res}^G_U\rho_\nu=\psi$, so $\left(\op{Res}^G_P\rho_\nu,\pi\right)=\left(\op{Res}^G_P\rho_\nu,\op{Ind}_U^P\psi\right)=\left(\op{Res}^G_U\rho_\nu,\psi\right)=1$. But $\dim\op{Res}^G_P\rho_\nu=\dim\rho_\nu=q-1=\dim\pi$, so we have equality.} Hence, $\rho_\nu$ is cuspidal by \Cref{prop:cuspidal-dim}.
		\item Let $\rho\coloneqq\rho_\nu$, $\rho'\coloneqq\rho_{\nu'}$, $j\coloneqq j_\nu$, and $j_{\nu'}\coloneqq j'$. If $\nu'$ is conjugate to $\nu$, then $j'$ is equal to $j$, as follows from definition \eqref{eq:def-cusp-j}. Hence, $\rho=\rho'$.

		Conversely, suppose that $\rho'$ is isomorphic to $\rho$. Then there exists an automorphism $\theta$ of $V$ (which is the vector space of all functions $f\colon K^\times\to\CC$) such that
		\begin{equation}
			\rho'(g)=\theta\rho(g)\theta^{-1}\qquad\text{for all }g\in G. \label{eq:def-theta-aut}
		\end{equation}
		In particular, \eqref{eq:def-theta-aut} is valid for every $g\in P$. However, $\op{Res}^G_P\rho=\pi=\op{Res}^G_P\rho'$, and $\pi$ is irreducible. Hence, by Schur's lemma, $\theta$ is multiplication by a scalar. In particular, $\theta$ commutes with every automorphism of $V$. Hence, \eqref{eq:def-theta-aut} implies that $\rho'(g)=\rho(g)$ for every $g\in G$.

		It follows that for every $\delta,y\in K^\times$ and $f\in V$,
		\[\nu'(\delta)f\left(\delta^{-1}y\right)=\left(\rho'\left(\begin{bmatrix}
			1 & 0 \\
			0 & \delta
		\end{bmatrix}\right)f\right)(y)=\left(\rho\left(\begin{bmatrix}
			1 & 0 \\
			0 & \delta
		\end{bmatrix}\right)f\right)(y)=\nu(\delta)f\left(\delta^{-1}y\right).\]
		Hence,
		\begin{equation}
			\nu'(\delta) = \nu(\delta) \qquad \text{for every }\delta\in K^\times. \label{eq:nup-eq-nu}
		\end{equation}
		Further, $\rho'(w')=\rho(w')$; hence,
		\[\sum_{x\in K^\times}\nu'\left(x^{-1}\right)j(yx)f(x)=\sum_{x\in K^\times}\nu\left(x^{-1}\right)j'(yx)f(x)\]
		for every $y\in K^\times$ and every $f\in V$. Using \eqref{eq:nup-eq-nu}, this implies that $j'(u)=j(u)$ for every $u\in K^\times$. This means that
		\begin{equation}
			\sum_{\op Nt=u}\psi\left(t+\overline t\right)\nu'(t)=\sum_{\op Nt=u}\psi\left(t+\overline t\right)\nu(t). \label{eq:eq-js}
		\end{equation}
		Let $\delta\in K^\times$ and replace the variable $t$ in \eqref{eq:eq-js} by $\delta t$. Using \eqref{eq:nup-eq-nu} to cancel $\nu(\delta)$ on both sides, we have
		\[\sum_{t\overline t=v}\psi\left(\delta(t+\overline t)\right)\nu'(t)=\sum_{t\overline t=v}\psi\left(\delta(t+\overline t)\right)\nu(t)\]
		for every $v,\delta\in K^\times$. This can be rewritten as
		\begin{equation}
			\sideset{}{'}\sum_{t\overline t=v}\left(\nu'(t)+\nu'(\overline t)-\nu(t)-\nu(\overline t)\right)\psi\left(\delta(t+\overline t)\right)=0, \label{eq:almost-full-nu-eq-nup}
		\end{equation}
		where the prime over the summation symbols indicates that for every $t\in L^\times$ such that $t\overline t=v$, only one pair out of the two $(t,\overline t),(\overline t,t)$ contributes a summand to the summation.

		Now let $x$ be an element of $K$. Then there exists a $t\in L$ such that $t+\overline t=x$ and $t\overline t=v$. This element is a solution to the quadratic equation $X^2-xX+v=0$. (We repeat that $L$ is the unique quadratic extension of $K$.) The other solution of the equation is obviously $\overline t$. The expression $a_x\coloneqq\nu'(t)+\nu'(\overline t)-\nu(t)-\nu(\overline t)$ is therefore well-defined, and \eqref{eq:almost-full-nu-eq-nup} can be rewritten as
		\[\sideset{}{'}\sum_{t\overline t=v}a_x\psi(\delta x)=0.\]
		If $x_1\ne x_2$, then the characters $\delta\mapsto\psi(\delta x_1)$ and $\delta\mapsto\psi(\delta x_2)$ of $K^+$ are distinct. Hence, by Artin's lemma, $a_x=0$ for every $x\in K$. This means that
		\[\nu'(t)+\nu'(\overline t)=\nu(t)+\nu(\overline t)\qquad\text{for all }t\in L^\times.\]
		In addition,
		\[\nu'(t)\nu'(\overline t)=\nu(t)\nu(\overline t)\qquad\text{for all }t\in L^\times\]
		by \eqref{eq:nup-eq-nu}. Hence, the pairs $\left(\nu'(t),\nu'(\overline t)\right)$, $\left(\nu(t),\nu(\overline t)\right)$ are the solutions of the same quadratic equation over $K$. Hence,
		\begin{equation}
			\left\{\nu'(t),\nu'(\overline t)\right\}=\left\{\nu(t),\nu(\overline t)\right\}\qquad\text{for all }t\in L^\times. \label{eq:pair-nu-eq-pair-nup}
		\end{equation}
		In particular, \eqref{eq:pair-nu-eq-pair-nup} is true for a generator $t_0$ of the cyclic group $L^\times$. Suppose, for example, that $\nu(t_0)=\nu'(t_0)$. Then $\nu(t)=\nu'(t)$ for every $t\in L^\times$. If $\nu'(t_0)=\nu(\overline{t_0})$, then $\nu'=\overline\nu$.

		We have therefore proved that $\nu'$ is conjugate to $\nu$ over $K$.
		\qedhere
	\end{listalph}
\end{proof}

At this point, we would like to indicate an interesting duality between conjugacy classes of $G$ on one hand and irreducible representations of $G$ on the other hand. The elements $\alpha$ of $K^\times$ correspond bijectively to the pair of conjugacy classes $(c_1(\alpha),c_2(\alpha))$ (see \cref{sec:gl-2-k-classes}), and there are $q-1$ of them. Dually, the characters $\mu_1$ of $K^\times$ correspond bijectively to pairs $\left(\rho_{(\mu_1,\mu_1)},\rho'_{(\mu_1,\mu_1)}\right)$ of irreducible representations of dimensions $q$ and $1$ respectively (see \Cref{thm:non-cusp-reps}), and there are $q-1$ of them. Further, the pairs of elements $\alpha,\beta$ of $K^\times$ with $\alpha\ne\beta$ correspond to the conjugacy classes $c_3(\alpha,\beta)$, and there are $\frac12(q-1)(q-2)$ of them. Dually, the pairs of characters $\mu_1,\mu_2$ of $K^\times$ with $\mu_1\ne\mu_2$ correspond to the irreducible representations $\rho_{(\mu_1,\mu_2)}$ of $G$ of dimension $q+1$, and there are also $\frac12(q-1)(q-2)$ of them. Finally, the elements $\lambda$ of $L^\times\setminus K^\times$ correspond to the conjugacy classes $c_4(\lambda)$, whereas the characters $\nu$ of $L^\times$ that do not come from characters of $K^\times$ (i.e., non-decomposable) correspond to the cuspidal representations $\rho_\nu$ of $G$. In both sets that are $\frac12\left(q^2-q\right)$ elements.

We summarize these data in the following table.
\[\begin{array}{c|c|c|c|c|c}
	\text{elements of }L^\times & \text{conj. classes} & \text{chars. of }L^\times,K^\times & \text{irr. repr. of }G & \text{dim. of repr.} & \text{no. of elements} \\\hline\hline
	\alpha \in K^\times & c_1(\alpha) & \mu_1\in X(K^\times) & \rho'_{(\mu_1,\mu_1)} & 1 & q-1 \\
	& c_2(\alpha) & & \rho_{(\mu_1,\mu_1)} & q  & q-1 \\\hline
	\alpha,\beta\in K^\times & c_3(\alpha,\beta) & \mu_1,\mu_2\in X\left(K^\times\right) & \rho_{(\mu_1,\mu_2)} & q+1 & \frac12(q-1)(q-2) \\
	\alpha\ne\beta & & \mu_1\ne\mu_2 &&& \\\hline
	\lambda\in L^\times\setminus K^\times & c_4(\lambda) & \nu\in X\left(L^\times\right)\setminus X\left(K^\times\right) & \rho_\nu & q-1 & \frac12\left(q^2-q\right)
\end{array}\]

\section{The small Weil group and the small reciprocity law} \label{sec:weil-group}
Let $F/E$ be a finite Galois extension. Its Galois group, $\op{Gal}(F/E)$ acts on the multiplicative group $F^\times$ of $F$. Denote by $W(F/E)\coloneqq\op{Gal}(F/E)\ltimes F^\times$ the semi-direct product of $\op{Gal}(F/E)$ by $F^\times$. It consists of all pairs $(x,\sigma)$ where $x\in F^\times$ and $\sigma\in\op{Gal}(F/E)$. Multiplication is given by the formula
\[(x,\sigma)\cdot(y,\tau)\coloneqq(x\cdot\sigma y,\sigma\tau);\]
the one element is $(1,1)$, and the inverse is given by
\[(x,\sigma)^{-1}\coloneqq\left(\sigma^{-1}x^{-1},\sigma^{-1}\right).\]
The map $x\mapsto(x,1)$ is an embedding of $F^\times$ in $W(F/E)$. We identify $F^\times$ with its image. Then $F^\times$ is normal in $W(F/E)$ and its index is equal to the degree $[F:E]$.

The group $W(F/E)$ is in general not abelian. A typical commutator is
\[(x,\sigma)(y,\tau)(x,\sigma)^{-1}(y,\tau)^{-1}=\left(x\cdot\sigma(y)\cdot\sigma\tau\sigma^{-1}\left(x^{-1}\right)\cdot\sigma\tau\sigma^{-1}\tau^{-1}\left(y^{-1}\right),\sigma\tau\sigma^{-1}\tau^{-1}\right).\]
If in addition $\op{Gal}(F/E)$ is abelian, then this formula simplifies to
\begin{equation}
	(x,\sigma)(y,\tau)(x,\sigma)^{-1}(y,\tau)^{-1}=\left(x\cdot\sigma(y)\cot\tau\left(x^{-1}\right)\cdot y,1\right). \label{eq:weil-commutator}
\end{equation}

We now restrict our attention to the case where $E=K$ is our field with $q$ elements and $F=L$ its unique quadratic extension. The Galois group $\op{Gal}(L/K)$ consists of two elements, a conjugation, the action of which is denoted by a bar, and the identity automorphism. In this case, $W(L/K)$ is called the small Weil group of the extension $L/K$. It is a finite group having $w(q-1)$ elements, and it can be described as the free group generated by $L^\times$ and with the relations
\begin{equation}
	\varphi^2=1\qquad\text{and}\qquad x\varphi=\varphi\overline x\text{ for }x\in L^\times. \label{eq:weil-relations}
\end{equation}
Now, $W(L/K)$ contains the abelian normal subgroup $L^\times$ of index $2$. Hence, its irreducible representations are of degree $\le2$.

We would like to establish a correspondence between the two-dimensional representations of $W(L/K)$ (not only the irreducible ones) and the higher-dimensional representations of $G$. As a first step toward this goal, let us compute the number of characters of $W(L/K)$. This number is equal to the index of the commutator subgroup $W(L/K)^c$ of $W(L/K)$. Indeed, applying \eqref{eq:weil-commutator} to the four possible pairs $(\sigma,\tau)$, one concludes that
\begin{equation}
	W(L/K)^c=\left\{z\overline z:z\in L^\times\right\}. \label{eq:all-weil-commutators}
\end{equation}
By \Cref{cor:h90}, the right-hand side is equal to $\left\{x\in L^\times:\op Nx=1\right\}$; hence, $\left[W(L/K):W(L/K)^c\right]=2(q-1)$ by \Cref{lem:n-epi}. We have therefore proved the following.
\begin{lemma}
	The group $W(L/K)$ has $2(q-1)$ characters.
\end{lemma}

Consider now a two-dimensional representation $\tau$ of $W(L/K)$ its restriction to $L^\times$ decomposes into a direct sum of two characters. Let $\nu$ be one of them.
\begin{lemma} \label{lem:res-weil-rep}
	Fix everything as above.
	\begin{listalph}
		\item If $\nu$ is non-decomposable, then $\op{Res}_{L^\times}\tau=\nu\oplus\overline\nu$.
		\item $\nu$ is non-decomposable if and only if $\tau$ is irreducible.
	\end{listalph}
\end{lemma}
\begin{proof}
	By construction, there exists a vector $0\ne v_1\in V_\tau$ such that
	\begin{equation}
		\tau(x)v_1=\nu(x)v_1\qquad\text{for all }x\in L^\times. \label{eq:eigen-of-tau}
	\end{equation}
	Let $v_1'\coloneqq\tau(\varphi)v_1$. Then the relation $x\varphi=\varphi\overline x$ implies
	\begin{equation}
		\tau(x)v_1'=\nu(\overline x)v_1'\qquad\text{for all }x\in L^\times. \label{eq:another-eigen-of-tau}
	\end{equation}
	Hence, $\overline\nu$ is also a component of $\op{Res}_{L^\times}\tau$. If $\nu\ne\overline\nu$, then $\op{Res}_{L^\times}\tau=\nu\oplus\overline\nu$, and (a) is proved.

	In order to prove (b), suppose first that $\tau=\tau_1\oplus\tau_2$ is reducible. Then the $\tau_i$ are characters of $W(L/K)$ and since $x\overline x^{-1}\in W(L/K)^c$ by \eqref{eq:all-weil-commutators}, we have $\tau_i(x)=\tau_i(\overline x)$. It follows that $\op{Res}_{L^\times}\tau_i$ are decomposable characters of $L^\times$. Also, $\nu$ must be equal to one of them; hence, $\nu$ is also decomposable.

	Conversely, suppose that $\nu=\overline\nu$. There are two possibilities. Either $\nu_1'=\nu(\varphi)v_1$ is a multiple of $v_1$, or $v_1$ and $v_1'$ are linearly independent. In the first case, $\nu\colon L^\times\cup\{\varphi\}\to\CC$ is a map which is multiplicative on $L^\times$ and agrees with the defining relations \eqref{eq:weil-relations} of $W(L/K)$. Hence, $\nu$ can be extended to a character of $W(L/K)$ that happens to be a component of $\tau$ by \eqref{eq:eigen-of-tau} and because $v_1'=\tau(\varphi)v_1$. This implies that $\tau$ is reducible.

	In the second case, $v_1$ and $v_1'$ generate $V_\tau$. Then \eqref{eq:eigen-of-tau} and \eqref{eq:another-eigen-of-tau} imply that $\tau(x)v=\nu(x)v$ for every $x\in L^\times$ and $v\in V_\tau$. Let $\nu_2$ be an eigenvector of $\tau(\varphi)$. Then because $\tau(x)v_2=\nu(x)v_2$, we prove as above that $v_2$ is an eigenvector of $W(L/K)$. This implies that $\tau$ is reducible.
\end{proof}

Now consider more closely the case where $\tau=\tau_1\oplus\tau_2$ is a reducible representation of $W(L/K)$. By \Cref{lem:res-weil-rep}, there exist characters $\mu_1$ and $\mu_2$ of $K^\times$ such that $\tau_i(x)=\mu_i(\op Nx)$ for $i=1,2$ and for every $x\in L^\times$. We may therefore write $\tau=\tau_{(\mu_1,\mu_2)}$. The pair $(\mu_1,\mu_2)$ defines a character $\mu$ of $B$ (by \eqref{eq:def-char-on-b}). If $\mu_1=\mu_2$, then $\widehat\mu=\rho_{(\mu_1,\mu_1)}\oplus\rho'_{(\mu_1,\mu_1)}$. We correspond $\tau$ to $\rho_{(\mu_1,\mu_1)}$. If $\mu_1\ne\mu_2$, then $\widehat\mu=\rho_{(\mu_1,\mu_2)}$. We correspond $\tau$ to $\rho_{(\mu_1,\mu_2)}$.

This correspondence is injective.\footnote{Piatetski-Shapiro claims that this correspondence is bijective, but the following description of the inverse map only hits reducible representations $\tau$ with $\tau(\varphi)=1$.} Indeed, starting from a pair of characters $(\mu_1,\mu_2)$ of $K^\times$, we define characters $\tau_1,\tau_2$ of $W(L/K)$ by $\tau_i(x)\coloneqq\mu_i(\op Nx)$ and $\tau_i(\varphi)=1$. This definition makes sense because it is compatible with the relations \eqref{eq:weil-relations}. Then $\tau_{(\mu_1,\mu_2)}=\tau_1\oplus\tau_2$ is a reducible two-dimensional representation of $W(L/K)$, and $\rho_{(\mu_1,\mu_2)}$ corresponds to $\tau_{(\mu_1,\mu_2)}$.

Finally, let $\nu$ be a non-decomposable character of $L^\times$. Define a two-dimensional representation $\tau_\nu$ of $W(L/K)$ by
\[\tau_\nu(x)\coloneqq\begin{bmatrix}
	\nu(x) & 0 \\
	0 & \nu(\overline x)
\end{bmatrix}\text{ for }x\in L^\times,\qquad\text{and}\qquad\tau_\nu(\varphi)\coloneqq\begin{bmatrix}
	0 & 1 \\
	1 & 0
\end{bmatrix}.\]
Then $\op{Res}_{L^\times}\tau_\nu=\nu\oplus\overline\nu$; hence, $\tau_\nu$ is irreducible. If $\nu'$ is an additional non-decomposable character of $L^\times$, then $\tau_\nu=\tau_{\nu'}$ if and only if $\nu'$ is conjugate to $\nu$. It follows that $W(L/K)$ has exactly $\frac12\left(q^2-q\right)$ irreducible representations of the form $\tau_\nu$. Taking into account \Cref{lem:res-weil-rep}, we deduce from the identity
\[2\left(q^2-1\right)=2^2\cdot\frac12\left(q^2-1\right)+1^2\cdot2(q-1)\]
that the $\tau_\nu$ are all the irreducible representations of $W(L/K)$.

Adding \Cref{prop:cusp-reps} to the above arguments, we have proved the following reciprocity law.
\begin{theorem}
	There exists an injective correspondence between the two-dimensional representations $\tau$ of the Weil group $W(L/K)$ and the higher-dimensional representations of $\op{GL}(2,K)$.

	Using the above notation, the correspondence may be described as follows.
	\begin{listalph}
		\item For reducible $\tau$: if $(\mu_1,\mu_2)$ is a pair of characters $K^\times$, then $\tau_{(\mu_1,\mu_2)}$ corresponds to $\rho_{(\mu_1,\mu_2)}$.
		\item For irreducible $\tau$: if $\nu$ is a non-decomposable character of $L^\times$, then $\tau_\nu$ corresponds to $\rho_\nu$.
	\end{listalph}
\end{theorem}

\end{document}